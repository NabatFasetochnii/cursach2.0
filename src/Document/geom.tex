\subsection{Координаты планеты}\label{subsec:coord}
Рассмотрим планету радиуса $R_p$ и массы $M_p$, вращающуюся вокруг звезды радиуса $R_*$ массой $M_*$.
Мы выбираем систему координат, центрированную на звезде, плоскость OXY совпадает с плоскостью неба,
ось Z направлена на наблюдателя.
Сонаправим ось OX с линией узлов, поместим нисходящий узел орбиты планеты вдоль положительного направления оси OX,
тогда $\Omega = 180^\circ$.

Расстояние между планетой и звездой задаётся уравнением:
\begin{gather}
    r = \frac{a\left( 1-e^2 \right)}{1+e \cos f},
\end{gather}
где $a$ --- большая полуось орбиты, $f$ --- истинная аномалия, $e$ --- эксцентриситет орбиты.

Выразим декартовы координаты планеты:
\begin{gather}
    X = -r \cos (\omega + f),\\
    Y = -r \sin (\omega + f) \cos i,\\
    Z =  r \sin (\omega + f) \sin i,
\end{gather}
где $i$ --- наклонение орбиты.

Планета находится в перицентре при минимальном значении $\sqrt {X^2+Y^2}$ и
в апоцентре при максимальном значении этого выражения, тогда

\begin{gather}
    \sqrt {X^2+Y^2}=\frac{a(1-e^2)}{1+e \cos f}\sqrt {1-\sin^2 (\omega +f)\sin^2 i},
\end{gather}

Исходя из этого, в зависимости от типа затмения можно сделать два приближения для истинной аномалии
в момент транзита~\cite{Kipping_2008}:

\begin{gather}
    f_1=\frac{\pi}{2}-\omega,\\
    f_2=-\frac{\pi}{2}-\omega,
\end{gather}
где $f_1$ характеризует первичное затмение, а $f_2$ --- вторичное.

\subsection{Большая полуось}\label{subsec:semi-major}
Зная орбитальный период планеты и спектральный класс звезды, благодаря третьему закону кеплера, мы можем
оценить большую полуось:
\begin{gather}
    \frac{a^3}{P^2}=\frac{G(M_*+M_p)}{4\pi^2},
\end{gather}
где $a$ --- большая полуось орбиты планеты, $P$~---~орбитальный период планеты.

Пренебрегая массой планеты и оценивая массу звезды из её спектрального класса получаем следующую
зависимость для большой полуоси:
\begin{gather}
    a\approx \sqrt[3]{GM_*\left( \frac{P}{2\pi} \right)^2}. \label{kepler}
\end{gather}

%\subsection{Орбитальная скорость}\label{subsec:orb_veliosity}
%В простейшем случае круговой орбиты орбитальная скорость является постоянной и задается
%\begin{gather}
%    v=\frac{2\pi a}{P}.
%\end{gather}

\subsection{Наклонение орбиты, прицельный параметр и время транзита}\label{subsec:mix}
Транзит имеет 4 ключевые точки: момент касания диска планеты с диском звезды,
когда планета ещё не затмевает звезду~---~вход в транзит,
моменты начала и конца плоской части кривой блеска~---~когда планета касается внутренней части края звезды
и момент касания диска планеты с диском звезды,
когда планета уже не затмевает звезду~---~выход из транзита.
Эти позиции проиллюстрированы на рис.~\ref{fig:points}.

Также, на рис.~\ref{fig:points} показан прицельный параметр b~---~наименьшее расстояние между центром
звезды и центром планеты в проекции на небо,
то есть расстояние между центрами звезды и планеты в середине транзита.
\begin{figure}[h!]
    \begin{center}
        \includegraphics[width=10cm]{3}
        \caption{Положение диска планеты относительно диска звезды в четырех точках контакта.}
        \label{fig:points}
    \end{center}
\end{figure}

Прицельный параметр связан с наклонением орбиты, i, следующим соотношением
\begin{gather}
    b=a \cos i.
\end{gather}
Однако, в некоторой литературе, как например в~\cite{Kipping_2008} или в~\cite{winn2014transits},
за прицельный параметр принята безразмерная величина $b=\frac{a \cos i}{R_*}$.
В случае эллиптичной орбиты

\begin{gather}
    b_1=a \cos i \left( \frac{1-e^2}{1+e \sin \omega} \right),\\
    b_2=a \cos i \left( \frac{1-e^2}{1-e \sin \omega} \right).
\end{gather}

Геометрия прицельного параметра и наклонение орбиты изображена на рис.~\ref{fig:impact}.
\begin{figure}[h!]
    \begin{center}
        \includegraphics[width=10cm]{1}
        \caption{Геометрия транзита с отличным от нуля прицельным параметром, b, или что тоже самое,
            отличным от 90\textdegree~наклонением.}
        \label{fig:impact}
    \end{center}
\end{figure}

Продолжительность транзита сильно зависит от прицельного параметра.
Так для круговой орбиты продолжительность транзита будет
\begin{gather}
    T_{tot}=P\frac{\alpha}{2\pi}, \label{tot}
\end{gather}
где $\alpha$~---~это длина дуги орбиты от первого касания до последнего.

На рис.~\ref{fig:time} показан прямоугольный треугольник, который имеет гипотенузу длины $R_* + R_p$ и
вертикальную сторону, равную прицельному параметру b, и, следовательно, имеет длину $a \cos i$.
Третья, горизонтальная, сторона треугольника соединяет положения центра диска планеты в середине транзита
и четвертого контакта.
По теореме Пифагора длина этой стороны:

\begin{gather}
    l=\sqrt{\left(R_*+R_p \right)^2-a^2\cos^2 i}.
\end{gather}

Эта величина есть лишь проекция на небо пути планеты, но в нашем случае это допустимое приближение.
\begin{figure}[h!]
    \begin{center}
        \includegraphics[width=10cm]{4}
        \caption{К расчёту продолжительности транзита в случае круговой орбиты.}
        \label{fig:time}
    \end{center}
\end{figure}

Затмения могут быть видны только определённым наблюдателям, которые видят орбиту планеты почти с ребра.
Следует заметить, что из рис.~\ref{fig:time} так же можно получить предельные значения прицельного параметра,
при которых транзит будет происходить

\begin{gather}
    0\leq a \cos i < R_*+R_p,\\
    0\leq \cos i < \frac{R_*+R_p}{a}\approx\frac{R_*}{a}.
\end{gather}
Но при таком предельном значении диск планеты не полностью будет заходить на диск звезды,
а значит будет особенным образом изменяться площадь скрываемой части диска звезды.
Полностью планета будет заходить на диск звезды при

\begin{gather}
    0\leq \cos i < \frac{R_*-R_p}{a}.
\end{gather}

Когда планета вращается вокруг своей звезды, ее тень описывает конус, который выметает полосу на небесной сфере,
Удаленный наблюдатель в пределах полосы тени увидит транзиты.
Угол раскрытия конуса, $\Theta$, удовлетворяет условию $\sin \Theta=( R_* + R_p)/r$,
где r - мгновенное расстояние между звездой и планетой.
Этот конус называется полутень.
Существует также внутренний конус --- область полной тени --- определяемая $\sin \Theta=( R_* - R_p)/r$,
внутри которого планета полностью заходит на диск звезды.

Угол $\alpha/2$ в уравнении~\ref{tot} можно приближённо считать как $\arcsin \frac{l}{a}$, тогда
\begin{gather}
    T_{tot}=\frac{P}{\pi}\arcsin\frac{l}{a}=\frac{P}{\pi}\arcsin \left(\frac{\sqrt{\left(R_*+R_p \right)^2-a^2\cos^2 i}}{a} \right)\\
    T_{tot} \approx \frac{P}{\pi}\arcsin \left(\sqrt{\frac{R_*^2}{a^2}-\cos^2 i} \right) \label{inc}
\end{gather}

Последняя выкладка справедлива при $a\gg R_*\gg R_p$.
Так как дуга орбиты между вторым и третьим касаниями меньше на два радиуса планеты,
то продолжительность плоской части транзита:

\begin{gather}
    T_{full}=\frac{P}{\pi}\arcsin \left(\frac{\sqrt{\left(R_*-R_p \right)^2-a^2\cos^2 i}}{a} \right)
\end{gather}

Приближение для эллиптической орбиты~\cite{winn2014transits}:

\begin{gather}
    T_{tot}=\frac{\sqrt{1-e^2}}{1\pm e \sin \omega }\frac{P}{\pi}\arcsin \left(\frac{\sqrt{\left(R_*+R_p \right)^2-a^2\cos^2 i}}{a} \right),\\
    T_{full}=\frac{\sqrt{1-e^2}}{1\pm e \sin \omega }\frac{P}{\pi}\arcsin \left(\frac{\sqrt{\left(R_*-R_p \right)^2-a^2\cos^2 i}}{a} \right),
\end{gather}
где "$+$"\, соответствует первичному транзиту, а "$-$"\, --- вторичному.

Если мы знаем спектральный тип звезды, мы можем вывести приблизительные значения $M_*$ и $R_*$, предположив,
что они типичны для спектрального типа.
Тогда продолжительность транзита и орбитальный период достаточны для вывода при помощи
уравнений~\ref{inc}~и~\ref{kepler} приблизительного наклона орбиты.

Время входа в транзит, $\tau_{i}$, и выхода из транзита, $\tau_{e}$, равны для круговой орбиты в силу симметрии задачи,
в отличие от эллиптической орбиты, для которой существует оценка различия этих времён~\cite{winn2014transits}:
\begin{gather}
    \frac{\tau_e-\tau_i}{\tau_e+\tau_i}\sim e \cos \omega \left( \frac{R_*}{a} \right)^3 \sqrt{\left( 1-(R_*b)^2 \right)^3}.
\end{gather}
Для близких к своей звезде планет, $R_*/a=0.2$, это выражение будет меньше, чем $10^{-2} e$,
когда же для более далёких планет оно значительно меньше.
Поэтому с хорошей точностью можно принять $\tau_i=\tau_e=\tau$ в общем случае.

