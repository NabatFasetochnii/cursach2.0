\chapter{Результаты и их обсуждение}
В ходе данной работы было обработано 33 спектра звезды HD~34078, полученных на эшелле-спектрографах высокого разрешения за период времени с 1997 года по 2015 год. Это значительно больший промежуток времени, по сравнению с работой \cite{Boisse2009}, в которой наблюдательные данные представлены лишь до 2007 года.

По результатам, полученным в ходе выполнения данной работы, можно сделать вывод о том, что нами были обнаружены вариации параметров линий поглощения СН и СН+, так как разброс их значений за исследованный период времени довольно велик. Проанализировав полученные данные, можно говорить о видимом уменьшении W линии СН+ и, соответственно, лучевой концентрации вещества, примерно на 40~\% за последние 18 лет (см. рисунки~\ref{pic:WCH} и~\ref{pic:NCH}). Причем наиболее интенсивный спад наблюдается примерно за последние четыре года. В то же время изменение W линии СН за исследованный промежуток времени происходило в пределах точности измерений (см. рисунки~\ref{pic:WCH} и~\ref{pic:NCH}). 
 
\begin{figure}[h]
\centering
\includegraphics[scale=0.5]{figures/WCH.eps}
\caption{Зависимость W линий СН~$\lambda4300$ и CH+~$\lambda4232$ от времени. Красным отмечены данные, для которых измерения W затруднены из-за относительно низкого отношения сигнала к шуму и блендирования линий
}
\label{pic:WCH}
\end{figure} 
\begin{figure}[h]
\centering
\includegraphics[scale=0.4]{figures/N.eps}
\caption{Зависимость N для линий СН~$\lambda4300$ и CH+~$\lambda4232$ от времени. Красным отмечены данные, для которых измерения W, а следовательно и N, затруднены из-за относительно низкого отношения сигнала к шуму и блендирования линий
}
\label{pic:NCH}
\end{figure} 

В работе \cite{Boisse2009} также были отмечены вариации W исследуемых линий поглощения, однако сама зависимость от времени имеет другой вид, по сравнению с зависимостью, полученной нами (см. приложение~\ref{Boisse}). При этом в работе \cite{Boisse2009} также использовались некоторые спектры, применявшиеся для оценок W и N в данной работе. Невозможно сказать точно, с чем связаны подобные различия, однако это не влияет на вывод о том, что соотношение W линий CН и СН+ заметно изменяется за  последние четыре года.

Так же наглядно можно подтвердить полученные результаты сравнивая профили линий поглощения за разные даты, полученные на спектрографах с одинаковым спектральным разрешением. На рисунке~\ref{pic:profiles} показано сравнение профилей линий СН+ и CH за разные даты. Как можно видеть, интенсивность линий СН+ значительно уменьшилась за период времени с 2005 год по 2012 год, в то время как профили линий СН остаются неизменными. 
\begin{figure}[h]
\centering
\includegraphics[scale=0.65]{figures/profiles.eps}
\caption{Сравнение профилей линий СН и СН+. Рисунок взят из работы~\cite{Jacek}}
\label{pic:profiles}
\end{figure} 

Наиболее интересные результаты были получены во время последних наблюдений на спектрографе UFES 22 января и 4  марта 2015 года. Наблюдается значительное уменьшение эквивалентной ширины линии СН+~$\lambda$4232, за эти даты примерно на 30~\%, в то время как изменение в значении эквивалентной ширины линии СН~$\lambda$4300 менее интенсивно. Причина данных изменений может быть связана с вариациями физических условий на небольших масштабах в окрестности звезды AE~Aur. В то время как вариации физических условий могут быть связаны с взаимодействием ударной волны и излучения, идущих от звезды, с достаточно плотными сгустками газа в ее окрестности. Это согласуется с выводами полученными в работе~\cite{Gratier2014}.