\chapter{Способы и методы теоретических и аналитических исследований}\label{ch:3}
%\chapter{Методика эксперимента}
%\chapter{Методика проведения наблюдений и измерений}
%\chapter{Способы и методы решения задачи}


\section{Наблюдательные данные}\label{sec:survey-data}
В данной работе для исследования межзвездных молекулярных линий поглощения использовались спектральные данные высокого разрешения, полученные на различных телескопах за период с 1997 по 2015 год. В общей сложности было обработано 33 спектра, из них:
\begin{asparaitem}
    \item 6 спектров, полученных на спектрографе BOES, установленном на 1.8-метровом рефлекторе в BOAO (Южная Корея);
    \item 21 спектр -- на спектрографе МАЭСТРО, установленном на телескопе Цейсс-2000 в Терскольской обсерватории (Россия);
    \item 1 спектр -- на спектрографе MIKE, установленном на 6.5-метровом Магеллановом телескопе в обсерватории Las Campanas (Чили);
    \item 1 спектр -- на спектрографе UVES, установленном на одном из 8.2-метровых телескопов в Европейской Южной Обсерватории (Чили);
    \item 1 спектр -- на спектрографе HARPS-N, установленном на 3.58-метровом TNG в FGG -- INAF (Испания);
    \item 3 спектра -- на спектрографе UFES, установленном на 1.2-метровом телескопе в АО УрФУ (Россия).
\end{asparaitem}


\section{Методики обработки спектральных данных}\label{sec:method-spec}
Первичная обработка спектральных данных, полученных на спектрографе UFES в АО УрФУ, выполнялась в программном пакете DECH, созданном сотрудником САО РАН Г.~А~Галазутдиновым~\cite{Galazutdinov1992} и включающим в себя программы Dech~95 (spectra imaging) и Dech~20t (spectra processing).
Dech~95 предназначен для просмотра изображения эшелле-спектрограммы и для работы с этим изображением, включающей в себя следующие этапы:
\begin{asparaitem}
    \item медианное усреднение всех изображений (объекта, торий-аргоновой лампы, снимка шума считывания, плоского поля), что позволяет повысить отношение сигнал к шуму и получить изображение, свободное от следов космических частиц;
    \item вычитание среднего кадра снимка шума считывания из всех изображений;
    \item построение маски -- таблицы с координатами положения спектральных порядков.
    Маска строится вручную путем расстановки реперных точек в центре спектральных порядков.
    Достаточно расставить точки на первых двух порядках;
    \item экстракция порядков -- извлечение одномерного спектра из изображения.
\end{asparaitem}

Дальнейшая обработка спектральных данных проводится с помощью программы Dech~20t, предназначенной для работы с экстрагированным спектром в отдельных порядках. Она позволяет:
\begin{asparaitem}
    \item удалять дефекты в спектре, например, плохие пиксели;
    \item сглаживать спектры;
    \item делить их на другие спектры;
    \item строить дисперсионную кривую для одного порядка и всего спектра. Подробное описание процесса построения дисперсионной кривой представлено в подразделе~\ref{subsec:dcm};
    \item измерять лучевые скорости;
    \item строить уровень непрерывного спектра (континуум). Более подробно построение уровня непрерывного спектра рассмотрено в подразделе~\ref{subsec:contin};
    \item измерять эквивалентные ширины линий несколькими способами (см. подраз-\\дел~\ref{subsec:contin});
    \item измерять лучевые концентрации вещества (см. раздел~\ref{sec:n}).
\end{asparaitem}

\subsection{Построение дисперсионной кривой}\label{subsec:dcm}
Построение дисперсионной кривой необходимо для перехода от шкалы в величинах пикселей к шкале длин волн.
Для этого нужно отождествить линии лабораторной (Th-Ar) лампы с линиями атласа для данного инструмента.

Для отождествленной линии ставится маркер и отмечается отсчет по длине волны.
Местоположение маркера определяется по максимальному совпадению профиля линии и его зеркального отражения.
При этом внутри порядка распределение этих маркеров должно идти по возможности максимально равномерно.
По всем маркерам внутри порядка строится средняя дисперсионная кривая.
Положения реперов аппроксимируются полиномом, степень которого задается вручную и не должна превышать значения (n-1), где n~--~количество расставленных маркеров.
Точность построения дисперсионной кривой характеризуется среднеквадратичным отклонением положения маркеров от этой кривой.
Результаты построения дисперсионной кривой для одного порядка представлены на рисунке~\ref{fig:dcm}.
В верхней части рисунка показаны реперы и форма дисперсионной кривой, в нижней~--~отклонения реперных точек от средней дисперсионной кривой, а в таблице представлена информация о каждом репере.
\begin{figure}[h]
    \centering
    \includegraphics[scale=0.65]{figures/dcm.eps}
    \caption{Результаты построения дисперсионной кривой для 55 спектрального порядка}
    \label{fig:dcm}
\end{figure}
Так как изображения звезды и лампы получены на том же приборе и при экстракции спектров используется одна и та же маска, то длина порядков и дисперсионная кривая для изображений звезды и лампы -- одинаковы.

Также программный пакет Dech~20t позволяет строить глобальную дисперсионную кривую для всех спектральных порядков.
Глобальная дисперсионная кривая может быть представлена функцией\eqref{eq:disp}), которая аналитически связывает реперы внутри одного спектрального порядка и сами порядки между собой.

\begin{equation}
    \lambda (x,m) = \sum_{i=0}^P \sum_{j=0}^O a_{ij} x^i m^j, \label{eq:disp}
\end{equation}
где $\lambda$~--~длина волны; \\ $x$~--~координата репера в пикселях; \\ $m$~--~номер порядка; \\ $a_{ij}$~--~коэффициент полинома (определяется методом наименьших квадратов); \\ $P$~--~степень полинома по $x$; \\ $O$~--~степень полинома по $m$.

Глобальная дисперсионная кривая обеспечивает надежную шкалу длин волн, даже в тех порядках, в которых не определены реперные точки.
Качество ее построения напрямую зависит от количества и точности расстановки реперов, а также от степени глобального полинома по пикселям (ByX) и по порядкам (ByM).
Для определения оптимального значения степеней полинома по пикселям и по порядку используется метод простых итераций, встроенный в программный пакет Dech~20t.

\subsection{Методика построения уровня непрерывного спектра и оценки~W в программном пакете Dech~20t}\label{subsec:contin}

Процесс построения уровня непрерывного спектра в программном пакете Dech~20t заключается в ручном расставлении реперных точек (не более 30), приблизительно на середине шумовой дорожки спектра, в местах свободных от линий излучения или поглощения. После чего реперные точки приближаются сплайном, который также должен проходить посередине шума и не искривляться над линией. Данный этап в обработке спектров является наиболее важным и трудоемким, так как ошибки в построении уровня непрерывного спектра напрямую влияют на вид профиля линии, а это, в свою очередь, непосредственно сказывается на точности определения эквивалентной ширины линии и лучевой концентрации вещества.
Нормализованный спектр используется для измерения W линий.
В программном пакете Dech~20t имеется три основных способа выполнения данных измерений:

\begin{asparaitem}
    \item прямое интегрирование.
    Данный способ является наиболее оптимальным и используется по умолчанию.
    Для этого необходимо указать левую и правую границы измеряемой линии, а также определить значение отношения сигнала к шуму, что позволяет оценить ошибки измерений.
    Значение W вычисляется
    по всем точкам профиля линии как:
    \begin{equation}
        W = \int(1-\frac{F_\lambda}{F_0})d\lambda \label{W},
    \end{equation}
    где $F_\lambda$ -- интенсивность точки профиля линии; \\ $F_0$ -- интенсивность континуума; \\ $\lambda$ -- длина волны.
    \item Аппроксимация гауссианой.
    Профиль линии в данном способе приближается вручную с помощью гауссианы, а вычисление значения W производится по формуле~\eqref{W}.
    \item Построение профиля линии вручную.
    В этом способе необходимо определить границы измеряемой линии, после чего следует расставить реперные точки (аналогично методу построения континуума), по которым будет построен профиль линии и измерено значение W по формуле~(\ref{W}).
\end{asparaitem}


\section{Автоматическое приближение уровня непрерывного спектра и оценка W}\label{sec:auto-w}
До сих пор остается нерешенной проблема точности построения уровня непрерывного спектра, так как большую роль в этом играет субъективный человеческий фактор. Как уже упоминалось ранее, неточности в построении уровня непрерывного спектра могут привести к большим ошибкам в оценках эквивалентных ширин линий и лучевой концентрации вещества.

\begin{figure}[h]
    \centering
    \includegraphics[scale=0.75]{figures/EW2.eps}
    \caption{Метод автоматического построения уровня непрерывного спектра на примере линии СН~$\lambda4300$. Синим нарисованы участки спектра, свободные от линии. Зеленым~--~профиль линии. Красным~--~уровень непрерывного спектра}
    \label{fig:continuum}
\end{figure}

Для возможного решения данной проблемы нами был разработан и протестирован метод автоматического построения уровня непрерывного спектра и измерения W, реализованный на языке программирования Python. Для построения континуума и определения W необходимо выбрать три окна. Два из трех окон представляют собой «чистый» континуум -- участки континуума по обе стороны от измеряемой линии свободные от каких-либо линий. В третьем окне определяются границы и профиль самой линии поглощения. Для единообразности измерений W за различные даты границы каждого окна брались фиксированной ширины.

Построение континуума основывается на полиномиальном приближении точек шумовой дорожки, степень полинома можно варьировать от 1 до 3 в зависимости от сложности профиля линии и количества точек в выбранных окнах.
На рисунке~\ref{fig:continuum} представлен пример построения уровня непрерывного спектра для линии СН~$\lambda4300$.

Данная методика позволяет значительно снизить человеческий фактор и тем самым улучшить точность полученных оценок W. Но полностью исключить человеческий фактор с помощью данной методики невозможно, так как необходимо контролировать процесс построения континуума, особенно в области над линией. В случае плохих спектров влияние шума может сильно искривлять уровень непрерывного спектра над линией (см. рисунок~\ref{fig:badcontinuum}). В таком случае, для улучшения точности оценок W, построение континуума выполнялось вручную в программном пакете Dech~20t, для дальнейшего использования уже нормализованного спектра в программе для автоматического построения континуума, в целях его уточнения.
\begin{figure}[h]
    \centering
    \includegraphics[scale=0.75]{figures/badcontinuum.eps}
    \caption{Метод автоматического построения уровня непрерывного спектра на примере линии СН+~$\lambda4232$. Синим нарисованы участки спектра, свободные от линии. Зеленым~--~профиль линии. Красным~--~уровень непрерывного спектра}
    \label{fig:badcontinuum}
\end{figure}

После нормализации спектра, используя формулу~\eqref{W}, вычислялось значение W линии.
Использование данного метода позволяет намного быстрее и точнее учитывать ошибки проведения уровня непрерывного спектра и их вклад в ошибки измерения W, вычисление которых представлено формулами~\eqref{eps}~\cite{Rollinde2003}.
\begin{equation}
    \begin{split}
        \epsilon_r &= \sigma_F * \delta\lambda * sqrt{N}, \\
        \epsilon_a &= \Delta\lambda_\text{line} * |1-F|, \\ \label{eps}
        \epsilon &= \sqrt{\epsilon_r^2 + \epsilon_a^2},
    \end{split}
\end{equation}
где $\epsilon_r$ -- ошибка, связанная с шумом; \\ $\epsilon_a$ -- ошибка, связанная с неопределенностью в уровне континуума; \\ $ \sigma_F$ -- среднеквадратическое отклонение нормализованного потока в континууме; \\ $\delta\lambda$ -- размер пикселя; \\ $N$ -- количество пикселей под линией; \\ $\Delta\lambda_\text{line}$ -- абсолютная ширина линии; \\ $F$ -- среднее значение нормализованного потока в континууме; \\ $\epsilon$ -- ошибка измерения W.

В данной работе использовались как автоматический, так и ручной методы построения уровня непрерывного спектра и измерений W. В таблице~\ref{tab:tab1} приведено сравнение результатов измерений, полученных с использованием каждого из методов.
В этой таблице $W^a$~--~значение, полученное с использованием метода автоматического построения уровня непрерывного спектра, $W$~--~значение, полученное с помощью программы Dech~20t, $dW$~--~ошибки измерений.
Из данной таблицы видно, что существенного различия в оценках W каждым из методов не наблюдается, то есть значения W для одной и той же линии совпадают в пределах $(1-3)\sigma$.
При этом точность измерения у метода автоматического построения уровня непрерывного спектра и оценки W значительно выше.
Оценки W и N, представленные далее по тексту, получены с помощью метода автоматического приближения континуума.
\begin{table} [h]
    \caption{Сравнение методов построения континуума и измерения эквивалентной ширины для линии СН~$\lambda4300$. }
    \label{tab:tab1}
    \begin{center}
        \begin{tabular}{|c|c|c|c|c|c|c|}
            \hline
            Эпоха наблюдения & Спектрограф & Разрешение & $ W$, м\AA & $ dW$, м\AA & $ W^a$, м\AA & $ dW^a$, м\AA \\\hline
            2453040,5        & BOES        & 90000      & 53,24      & 0,53        & 52,65        & 0,36          \\\hline
            2453479,5        & МАЭСТРО     & 120000     & 52,82      & 1,87        & 53,97        & 0,75          \\\hline
            2454500,5        & BOES        & 90000      & 56,43      & 0,49        & 55,25        & 0,46          \\\hline
            2455554,5        & МАЭСТРО     & 40000      & 49,48      & 1,25        & 51,34        & 0,88          \\\hline
            2455924,5        & МАЭСТРО     & 40000      & 48,08      & 1,71        & 51,13        & 1,81          \\\hline
            2455938,5        & MIKE        & 40000      & 52.41      & 0,97        & 50,80        & 0,37          \\\hline
            2456611,5        & HARPS--N    & 115000     & 53,02      & 1,36        & 51,38        & 0,74          \\\hline
        \end{tabular}
    \end{center}
\end{table}

Так же немаловажно отметить, что в ходе измерений могут возникнуть сложности в определении границ линий, так как они могут быть блендированы звездными линиями поглощения или излучения. В качестве примера такого случая можно рассмотреть линию СН+~$\lambda4232$, которая блендирована линией звездного NeII (см. рисунок~\ref{fig:chne}).
\begin{figure}[h]
    \centering
    \includegraphics[scale=0.5]{figures/CHNe.eps}
    \caption{Межзвездная молекулярная линия СН+~$\lambda4232$ и звездная линия NeII. Сплошной линией изображен наблюдаемый спектр, пунктирной~--~теоретический спектр звезды}
    \label{fig:chne}
\end{figure}

Для исключения влияния звездной линии, было использовано несколько способов:
\begin{asparaitem}
    \item cглаживание вручную.
    Производилось непосредственно в программном пакете Dech~20t при помощи расставления реперных точек уровня непрерывного спектра.
    Также Dech~20t позволяет разбивать профиль линии на несколько компонент гауссианами, однако мы не использовали данный метод.
    Следует отметить, что подобные методы могут привести к большим ошибкам в точности измерений, так как профиль звездной линии может быть весьма сложен.\label{method}
    \item расчет W линии NeII на основе теоретических спектров.
    Ранее нами были определены основные физические параметры для данного объекта на основе моделирования звездных атмосфер.
    Уже имеющиеся данные об AE~Aur были использованы для определения обилия элемента в атмосфере и получения теоретического спектра наилучшим образом соответствующего наблюдаемому спектру.
    Затем, с помощью методов, описанных выше, было определено значение W звездной линии поглощения в теоретическом спектре, которое в дальнейшем вычиталось из значения W, полученного для межзвездной и звездной линий поглощения вместе.
\end{asparaitem}

На практике было выявлено, что большого различия между результатами, полученными обоими способами, не наблюдается (см.
таблицу~\ref{tab:tab2}).
Однако использование метода сглаживания вручную, как уже упоминалось ранее, может привести к большому разбросу значений W линии, из-за неточности расстановки реперных точек при построении уровня непрерывного спектра вручную.
\begin{table}[h]
    \caption{Сравнение методов исключения звездных линий поглощения.}
    \label{tab:tab2}
    \begin{center}
        \begin{tabular}{|c|c|c|c|c|c|c|}
            \hline
            Эпоха наблюдения & Спектрограф & Разрешение & $ W$, м\AA & $ dW$, м\AA & $ W^a$, м\AA & $ dW^a$, м\AA \\\hline
            2453040,5        & BOES        & 90000      & 41,83      & 1,24        & 42,79        & 0,59          \\\hline
            2453479,5        & МАЭСТРО     & 120000     & 43,67      & 0,92        & 44,34        & 1,28          \\\hline
            2454500,5        & BOES        & 90000      & 47,04      & 0,73        & 45,78        & 0,63          \\\hline
            2455554,5        & МАЭСТРО     & 40000      & 45,53      & 0,44        & 46,34        & 1,65          \\\hline
            2455924,5        & МАЭСТРО     & 40000      & 39,00      & 2,37        & 37,13        & 0,83          \\\hline
            2455938,5        & MIKE        & 40000      & 37,06      & 0,41        & 37,72        & 0,26          \\\hline
            2456611,5        & HARPS--N    & 115000     & 35,28      & 1,02        & 33,84        & 0,86          \\\hline
        \end{tabular}
    \end{center}
\end{table}


\section{Оценка лучевой концентрации вещества}\label{sec:n}
Оценки лучевой концентрации вещества производятся в рассмотрении оптически тонкого случая, то есть оптическая толщина линий не превышает единицу. Авторами работы \cite{Boisse2009} было показано, что приближение малой оптической толщины применимо для исследуемых линий. В частности об этом свидетельствует то, что кривая роста для этих линий может быть аппроксимирована простой зависимостью \cite{Boisse2009}:
\begin{equation}
    \begin{split}
        N(CH) &= 2,42*10^{11}*W^{1,48} \\
        N(CH+) &= 2,29*10^{11}*W^{1,48}\label{n}
    \end{split}
\end{equation}

Для перехода от величины W к лучевой концентрации нами использовалась зависимость~\eqref{n}.
В таблице~\ref{tab:tabn} представлены некоторые значения оценки лучевой концентрации.
Оценки ошибок N были получены с помощью программного пакета Dech~20t.
\begin{table}[h]
    \caption{Оценки лучевой концентрации вещества для линии СН~$\lambda4300$.}
    \label{tab:tabn}
    \begin{center}
        \begin{tabular}{|c|c|c|c|c|}
            \hline
            Эпоха наблюдения & Спектрограф & Разрешение & $ N, 10^{13}$ см$^{-2} $ & $ dN, 10^{13}$ см$^{-2} $ \\\hline
            2453040,5        & BOES        & 90000      & 8,54                     & 0,17                      \\\hline
            2453479,5        & МАЭСТРО     & 120000     & 8,86                     & 0,25                      \\\hline
            2454500,5        & BOES        & 90000      & 9,17                     & 0,85                      \\\hline
            2455554,5        & МАЭСТРО     & 40000      & 8,23                     & 0,36                      \\\hline
            2455924,5        & МАЭСТРО     & 40000      & 8,18                     & 0,44                      \\\hline
            2455938,5        & MIKE        & 40000      & 8,10                     & 0,05                      \\\hline
            2456611,5        & HARPS--N    & 115000     & 8,24                     & 0,22                      \\\hline
        \end{tabular}
    \end{center}
\end{table}