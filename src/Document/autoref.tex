\documentclass[autoref,subf,href,facsimile]{disser8}

\usepackage[a4paper,nohead,includefoot,mag=1000, margin=2cm,footskip=5mm]{geometry}
\usepackage[T2A]{fontenc}
\usepackage[utf8]{inputenc}
\usepackage{type1ec}
\usepackage[english,russian]{babel}
\usepackage{cmap}
\usepackage{paralist}
\ifpdf\usepackage{epstopdf}\fi

% Путь к файлам с иллюстрациями
%\graphicspath{{figires/}}

\begin{document}
    \singlespacing
% Включение файла с общим текстом диссертации и автореферата
% (текст титульного листа и характеристика работы).
    %%%%%% Общие поля титульного листа %%%%%%
\institution{Министерство науки и высшего образования Российской Федерации\\
ФГАОУ ВО <<УрФУ имени первого Президента России Б.Н. Ельцина>>\\
Институт естественных наук и математики}

%\department{наук о Земле} % Название департамента
\chair{астрономии, геодезии, экологии и мониторинга окружающей среды} % Название кафедры
\topic{Исследование экзопланетной системы TIC229510866\\
по данным телескопа МАСТЕР-Урал и архива ExoFOP-TESS} % Тема
%\topic{\Large{\textbf{Исследование экзопланетной системы TIC229510866 по
%данным телескопа МАСТЕР-Урал и архива ExoFOP-TESS}}}
\city{Екатеринбург} % Город
\date{\number\year} % Год


%%%%%% Титульный лист курсовой работы %%%%%%

\title{ОТЧЕТ\\ о курсовой работе} % Тип
\groupnumber{МЕН--381301} % Номер группы
\author{Чазов Никита Андреевич} % ФИО автора в именительном падеже

%%%%%% Титульный лист бакалаврской работы  %%%%%%
%\spec{21.03.03} %Шифр направления
%\course{Геодезия и дистанционное зондирование} % Название специальности или направления
%\profile{Космическая геодезия и навигация} %Название образовательной программы (указывается, если для направления несколько ОП)
%\apname{д.ф.-м.~н., доц. Э.~Д.~Кузнецов} % Имя лица, допускающего к защите (зав. кафедрой)
%\control{к.ф.-м.~н., асс. Е.~А.~Аввакумова} %Нормоконтролер
%\title{Выпускная квалификационная\\ работа бакалавра} % Тип
%\author{Иванова Ивана Ивановича} % ФИО автора в родительном падеже
%\sa{инженер-исследователь~АО~УрФУ С.~Ю.~Парфёнов} % Научный руководитель 
%\sasecond{к.ф.-м.~н., зав.~отделом физики Солнца А.~М.~Соболев} %Второй научный руководитель

%%%%%% Титульный лист дипломной работы  %%%%%%
%\spec{03.05.01} %Шифр специальности
%\speciality{Астрономия} % Специальность
%\apname{д.ф.-м.~н., доц. Э.~Д.~Кузнецов} % Имя лица, допускающего к защите (зав. кафедрой или директор департамента)
%\control{к.ф.-м.~н., асс. Е.~А.~Аввакумова} %Нормоконтролер
%\title{Дипломная работа} % Тип
%\author{Иванова Ивана Ивановича} % ФИО автора в родительном падеже
%\sa{инженер-исследователь~АО~УрФУ С.~Ю.~Парфёнов} % Научный руководитель 
%\sasecond{к.ф.-м.~н., зав.~отделом физики Солнца А.~М.~Соболев} %Второй научный руководитель

%%%%%% Титульный лист магистерской диссертации  %%%%%%
%\spec{03.05.01} %Шифр специальности или направления
%\masterprog{Астрофизика} % Название магистерской программы
%\apname{д.ф.-м.~н., доц. Э.~Д.~Кузнецов} % Имя лица, допускающего к защите (зав. кафедрой или директор департамента)
%\control{к.ф.-м.~н., асс. Е.~А.~Аввакумова} %Нормоконтролер
%\title{Магистерская диссертация} % Тип
%\author{Иванова Ивана Ивановича} % ФИО автора в родительном падеже
%\sa{инженер-исследователь~АО~УрФУ С.~Ю.~Парфёнов} % Научный руководитель 
%\sasecond{к.ф.-м.~н., зав.~отделом физики Солнца А.~М.~Соболев} %Второй научный руководитель



    \institution{}
    \topic{Название}

    \title{АВТОРЕФЕРАТ\\
    диссертации на соискание ученой степени\\
    кандидата физико-математических наук}

    \maketitle

% Внутренняя сторона обложки
    \noindent
    Работа выполнена в Федеральном государственном автономном образовательном учреждении высшего профессионального образования ''Уральский федеральный университет имени первого Президента России Б.~Н. Ельцина''

    \vspace{8mm}
    \noindent
    \textbf{Научный руководитель:}\\
    кандидат физико-математических наук, доцент по специальности, \\
    ЛОКТИН Александр Васильевич\\[8mm]
    \textbf{Официальные оппоненты:}\\
    БОБЫЛЕВ Вадим Вадимович,\\
    доктор физико-математических наук, заведующий лабораторией динамики Галактики Главной (Пулковской) астрономической обсерватории Российской академии наук\\[8mm]
    АНТОХИНА Элеонора Артуровна,\\
    кандидат физико-математических наук, старший научный сотрудник отдела звездной астрофизики Государственного астрономического института\\ им.~П.~К. Штернберга Московского государственного университета\\ им.~М.~В. Ломоносова\\[8mm]
    \textbf{Ведущая организация:}\\
    Федеральное государственное бюджетное учреждение науки Институт астрономии Российской академии наук\\[8mm]
    Защита состоится 4 октября 2013 года в 11 час. 30 мин.
    на заседании диссертационного совета Д.002.120.01 Главной  (Пулковской) астрономической обсерватории Российской академии наук (ГАО РАН), по адресу:\\ 196140, Санкт-Петербург, Пулковское шоссе, дом 65.

    \vspace{8mm}
    \noindent
    С диссертацией можно ознакомиться в библиотеке ГАО РАН.

    \vspace{8mm}
    \noindent
    Автореферат разослан 29 августа 2013 года.

%\vskip2ex\noindent
%Отзывы и замечания по автореферату в двух экземплярах, заверенные
%печатью, просьба высылать по вышеуказанному адресу на имя ученого секретаря
%диссертационного совета.
    \vspace{8mm}\noindent
    Ученый секретарь\\
    диссертационного совета
    \hfill
    \makeatletter
% вставка файла, содержащего факсимиле ученого секретаря
    \ifDis@facsimile
    \raisebox{-10mm}{\includegraphics[width=2cm]{sec-facsimile}}\hfill
    \fi%
    \makeatother
    Милецкий Евгений Викторович

    \clearpage

    \nsection{Общая характеристика работы}

% Актуальность работы
    \actualitysection
    \actualitytext


% Цель диссертационной работы
    \objectivesection
    \objectivetext

% Научная новизна
    \noveltysection
    \noveltytext

% Практическая значимость
    \valuesection
    \valuetext

% Результаты и положения, выносимые на защиту
    \resultssection
    \resultstext

% Апробация работы
    \approbationsection
    \approbationtext


% Личный вклад автора
    \contribsection
    \contributiontext


% Структура и объем диссертации
    \newpage
    \structtext


    \vspace{15mm}

%Благодарности
    \acknowledgementsection
    \acknowledgementtext

% ----------------------------------------------------------------
    \newpage
    \renewcommand{\bibnumfmt}[1]{#1*}
    \renewcommand\bibsection{\nchapter{Список публикаций}}

    \begin{thebibliography}{20}

    \end{thebibliography}

    \newpage
    \renewcommand{\bibnumfmt}[1]{#1}
    \renewcommand\bibsection{\nchapter{Список использованных источников}}
    \bibliographystyle{Kourovkastyle}
    \bibliography{Avvakumova}
% ----------------------------------------------------------------

\end{document}
