Звезды главной последовательности не имеют твердых поверхностей: их внешние слои состоят из плазмы.
Свет, исходящий от них, исходит из различных глубин звездной атмосферы.
Вероятность выхода фотона, испускаемого в определенном слое атмосферы, зависит от оптической толщины этого слоя.
При заданной частоте, $\nu$, оптическая толщина, $\tau_{\nu}$, является интегралом объёмного показателя ослабления среды,
$k_{\nu}$, умноженным на плотность, $\rho(s)$, вдоль пути, пройденного лучом света
\begin{gather}
    \tau_{\nu} = \int \rho(s) k_{\nu} ds
\end{gather}

Оптическая толщина зависит от частоты излучения, поэтому определенная физическая глубина в атмосфере звезды будет иметь
различную оптическую толщину, в зависимости от рассматриваемой частоты излучения.
Интенсивность света, прошедшего через среду, уменьшается экспоненциально:

\begin{gather}
    \frac{I}{I_0}=e^{-\tau_{\nu}},
\end{gather}
где $I$ --- интенсивность прошедшего света, а $I_0$ --- начальная интенсивность.

Фотон, выходящий из центра звездного диска, движется радиально наружу через звездную атмосферу.
Фотон, появляющийся из любого другого места на звездном диске,
движется под углом $\gamma$ к внешне направленному радиус-вектору.
Для фотонов, испускаемых на глубине h, длина пути через атмосферу звезды равна

\begin{gather}
    s \approx \frac{h}{\cos \gamma} = \frac{h}{\mu},
\end{gather}
где $\mu = \cos \gamma$, мы используем приближение $h \ll R_*$.

Оптическая толщина для данной физической глубины увеличивается по направлению к краю звезды.
Следовательно, звездный диск постепенно тускнеет по направлению к краю.
Более короткая длина пути, пройденного через атмосферу фотонами, движущимися радиально наружу, также означает,
что больше фотонов выходит из более глубоких слоев атмосферы в центре диска.
Как правило, более глубокие слои более горячие, и излучение, испускаемое там приближается к более синему спектру черного тела.
Это означает, что диск постепенно краснеет по направлению к краю, что укореняет нелинейную зависимость потемнения от
длины волны.
Зависимость потемнения звездного края от длины волны сразу же проявляется на многоцветных кривых блеска:
кривая блеска в фильтрах с большой длиной волны, в красной части спектра, имеет более плавно изгибающийся транзитный минимум,
когда же в фильтрах с короткой длиной волны, в синей части спектра, кривая блеска может быть настолько изогнуто,
что нельзя визуально определить точки второго и третьего касания~\cite{Haswell2010}.
Без учёта этого эффекта нельзя получить точные значения параметров системы~\cite{Espinoza}.

\subsection{Однородный диск}\label{subsec:disc}

Существует несколько стандартных моделей потемнения к краю, большая часть которых описываются в работе Мендела и Агола~\cite{Mandel_2002}.
В частности модель однородного диска (Uniform model) описывает транзит как затмение сферической звезды непрозрачной темной сферой,
то есть эта модель не учитывает потемнение к краю и опирается только на геометрические предпосылки задачи.
Для однородного источника отношение затененного к полному потоку равно

\begin{gather}
    F(p,z) = 1 - \lambda (p, z), \\
    \lambda (p, z) =
    \begin{cases}
        0                                                                                                        , & 1+p<z, \\
        \frac{1}{\pi} \left( p^2 k_0 +k_1 - \frac{\sqrt{4 z^{2} - \left( 1+z^{2} - p^{2} \right)^{2}}}{2} \right), & \abs{1-p} < z \leq 1+p, \\
        p^2                                                                                                      , & z \leq 1-p, \\
        1                                                                                                        , & z \leq p-1,
    \end{cases}
\end{gather}
где $d$ --- расстояние между центрами тел, $z=d/R_*$, $p=R_p/R_*$~---~отношение радиусов,
$k_1=\arccos \left[\left( 1-p^2 +z^2 \right) /2z \right]$, $k_0=\arccos \left[ \left( p^2 +z^2 - 1 \right) /2pz \right]$.

\subsection{Quadratic model}\label{subsec:quadratic-model}
Эта модель, так же как и предыдущая, представлена в статье~\cite{Mandel_2002}.
Она рассматривает функцию интенсивности диска звезды для заданной длины волны как квадратичный закон относительно $\mu$:
\begin{gather}
    I(r)=1-\gamma_1 (1-\mu) - \gamma_2 (1-\mu)^2 ,
\end{gather}
где $\gamma_1$ и $\gamma_2$ --- некоторые коэффициенты, характеризующие потемнение к краю, $\gamma_1+\gamma_2<1$.
Впервые такая функция была предложена Копалом~\cite{Kopal} в 1950 году для работы с тесными двойными системами.

Квадратичная модель является продолжением нелинейной модели Кларета~\cite{Claret_2000}.
Мендел и Агол модифицировали её, что позволило перейти от гипергеометрических функций к
более лёгким в вычислении эллиптическим интегралам.

\subsection{QPower2 model}\label{subsec:qpower2-model}
Макстед и Гилл воспользовались~\cite{Maxted_2019} законом с двумя коэффициентами power-2,
предложенным Хестрофером~\cite{Hestroffer}, поскольку тот превосходит прочие законы с двумя коэффициентами,
принятых в литературе в большинстве случаев, в особенности для холодных звёзд~\cite{Morello_2017}.
Функция интенсивности для заданной длины волны выглядит следующим образом
\begin{gather}
    I=1-c (1-\mu ^\alpha),
\end{gather}
где $\mu$ косинус угла между нормалью к поверхности и лучом зрения.

Использование показателя $\mu$, а не коэффициента некоторой степени $\mu$,
позволяет этому закону точно соответствовать форме профиля потемнения к краю звезды,
используя только один дополнительный параметр, в отличие от линейного закона потемнения к краю.

Кривая блеска вычисляется с помощью численного интегрирования.
Время, необходимое для выполнения численного интегрирования отдельного объекта, достаточно мало.
Но, если цель состоит в обнаружении и анализе транзитов в большом количестве высокоточных кривых блеска из таких исследований,
как Kepler, K2 или TESS, время вычислений может быть ограничивающим фактором.
Однако QPower2 позволяет перенести вычисления кривых блеска многих объектов на GPU, что делает данный
алгоритм привлекательным для массового анализа кривых блеска.

%\subsection{Общая модель}\label{subsec:general-model}
%%~\cite{g_2006}.
%
%Представлены простые в использовании аналитические формулы для расчета световых кривых внесолнечных планетарных транзитов.
%Уравнения являются функцией дробных радиусов планеты и родительской звезды,
%наклона орбиты и коэффициентов затемнения конечностей звезды.
%Кривые освещенности могут быть решены для этих параметров в зависимости от точности имеющихся наблюдений.
%Когда кривая радиальной скорости также доступна, как это обычно бывает для обеспечения природы системы,
%можно определить массы, радиусы и среднюю плотность как звезды, так и планеты.
%Уравнения справедливы для любой степени затемнения конечностей, а также для любого типа транзита.
%Случаи эксцентрических орбит, третьего света или
%ненулевой относительной светимости планеты могут быть легко приняты во внимание.
%Основное предположение состоит в том, что проекции как звезды,
%так и планеты на плоскость неба хорошо представлены круглыми дисками.
%Также обсуждаются последствия в случае, если это предположение неверно.
%Показаны практические применения, начиная с кривой освещенности фотометрически обнаруженной планеты OGLE-TR-113,
%полученной с помощью наземного телескопа.
%В качестве второго примера приведены результаты исследования кривой освещенности,
%полученной для прохождения гигантской планеты в HD 209458 с помощью космического телескопа Хаббла.
%Кратко обсуждаются процедуры для получения наилучших параметров соответствия.