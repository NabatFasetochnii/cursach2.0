\documentclass[a4paper, 12pt, subf, href, coursework]{disser8}
%% master -- for master thesis
%% bachelor -- for bachelor thesis
%% specialist -- for specialist diploma
%% coursework -- for course work

\usepackage{multirow} %объединение рядов по вертикале в таблицах
%\usepackage{cmap} %делает возможным поиск и выделение символов в файле pdf
\usepackage{longtable} %длинные таблицы
%\usepackage{array} % матрицы
\usepackage{paralist} %пакет для невложенных нумерованных и ненумерованных списков, оформленных по требованиям госта
\usepackage{enumitem} %если списки вложенные, то нужен вот этот пакет и три следующие за ним команды
\AddEnumerateCounter{\asbuk}{\@asbuk}{щ}
\setlist[1]{leftmargin=0cm,itemindent=\parindent}
\setlist[2]{leftmargin=\parindent,listparindent=\parindent,itemindent=\parindent}
%\DeclareGraphicsExtensions{.pdf,.png,.jpg}
\usepackage{epstopdf} % пакет для автоконвертации eps и ps картинок в pdf
\epstopdfsetup{update,suffix=,prefersuffix=false}
\AppendGraphicsExtensions{.ps}

\usepackage[paperwidth=21cm,paperheight=29.7cm,left=3cm,right=1.5cm,top=2cm,bottom=2cm]{geometry} % определение макета страницы
\usepackage[T2A]{fontenc}
\usepackage[utf8]{inputenc} % кодировка utf-8
\usepackage[english, russian]{babel} % переносы для русского и английского языков
%\usepackage{subfig} %пакет для картинок в картинке
\usepackage{wasysym} %всякие символы
%\usepackage{caption} %подписи к таблицам и рисункам
\captionsetup[figure]{labelsep=endash,justification=justified,singlelinecheck=true,font=small}
\captionsetup[table]{labelsep=endash,justification=justified,singlelinecheck=false,font=normalsize}

%%%%%%%%%
\usepackage{showkeys} %не забыть убрать в итоговой версии. Пакет для показа ссылок
%%%%%%%%%
%\usepackage{hyperref}
%\usepackage{amsmath} %делает гиперссылки из ссылок
\usepackage{mathtools}
\usepackage{icomma} %умная запятая
\usepackage{rmathbr} %пакет для переноса знака в конце строки
\usepackage[normalem]{ulem}
\usepackage{graphics} % картинки
\usepackage{booktabs}
\usepackage{textcomp} %пакет для того, чтобы безболезненно подчёркивать и переносить текст по строчкам
\usepackage{eucal}[mathscr]

%%%%%%%%%%%%%%%%%%%%%%%%%%%%%%%%%%%%%%%%%%%%%%%%%%%%%%%%%%%%%%%%%%%%%%%%%%%%%%%%%%%%%%%%%%%%%%%%%%%%%%%%%%%%%%%%
%\usepackage{lastpage} %пакет для того, чтобы узнать номер последней страницы
%\usepackage{totcount} %пакет, чтобы считать штуки
%
%\regtotcounter{page} %страниц
%\regtotcounter{figure} %рисунков
%
%\newtotcounter{citnum} %источников
%\def\oldbibitem{}
%\let\oldbibitem=\bibitem
%\def\bibitem{\stepcounter{citnum}\oldbibitem}
%
%\newtotcounter{longtabcnt}
%\def\oldltab{} \let\oldltab=\longtable
%\def\longtable{\stepcounter{longtabcnt}\oldltab}
%
%\newtotcounter{foofigure}
%\makeatletter
%\renewenvironment{figure}[1][\fps@figure]{
%\edef\@tempa{\noexpand\@float{figure}[1]}
%\@tempa
%\addtocounter{foofigure}{1}
%}{
%\end@float
%}
%\makeatother
%
%\newtotcounter{footable}
%\makeatletter
%\renewenvironment{table}[1][\fps@table]{
%\edef\@tempa{\noexpand\@float{table}[1]}
%\@tempa
%\addtocounter{footable}{1}
%}{
%\end@float
%}
%\makeatother
%
%\newcounter{fulltabcnt}
%
%
%
%\numberwithin{equation}{section}
%\numberwithin{figure}{section}
%\numberwithin{table}{section}


%%%%%%%%%%%%%%%%%%%%%%%%%%%%%%%%%%%%%%%%%%%%%%%%%%%%%%%%%%%%%%%%%%%%%%%%%%%

% правильная нумерация источников в списке литературы
\makeatletter
\renewcommand{\@biblabel}[1]{#1}
\makeatother
\newcommand*{\pd}[3][]{\ensuremath{\frac{\partial^{#1} #2}{\partial #3}}}
%нумерация разделов приложения заглавными русскими буквами
\renewcommand\theappendix{\Asbuk{chapter}}
\setcounter{tocdepth}{3}
\renewcommand{\topfraction}{0.85}
\renewcommand{\textfraction}{0.1}
\renewcommand{\floatpagefraction}{0.75}
\renewcommand{\theenumi}{\asbuk{enumi}}
\renewcommand{\theenumii}{\arabic{enumii}}

%%%%%%%%%%%%%%%%%%%%%%%%%%%%%%%%%%%%%%%%%%%%%%%%
\usepackage{physics}



\graphicspath{ {../figures/} }

\begin{document}
%    \nocite{*}

    %%%%%% Общие поля титульного листа %%%%%%
\institution{Министерство науки и высшего образования Российской Федерации\\
ФГАОУ ВО <<УрФУ имени первого Президента России Б.Н. Ельцина>>\\
Институт естественных наук и математики}

%\department{наук о Земле} % Название департамента
\chair{астрономии, геодезии, экологии и мониторинга окружающей среды} % Название кафедры
\topic{Исследование экзопланетной системы TIC229510866\\
по данным телескопа МАСТЕР-Урал и архива ExoFOP-TESS} % Тема
%\topic{\Large{\textbf{Исследование экзопланетной системы TIC229510866 по
%данным телескопа МАСТЕР-Урал и архива ExoFOP-TESS}}}
\city{Екатеринбург} % Город
\date{\number\year} % Год


%%%%%% Титульный лист курсовой работы %%%%%%

\title{ОТЧЕТ\\ о курсовой работе} % Тип
\groupnumber{МЕН--381301} % Номер группы
\author{Чазов Никита Андреевич} % ФИО автора в именительном падеже

%%%%%% Титульный лист бакалаврской работы  %%%%%%
%\spec{21.03.03} %Шифр направления
%\course{Геодезия и дистанционное зондирование} % Название специальности или направления
%\profile{Космическая геодезия и навигация} %Название образовательной программы (указывается, если для направления несколько ОП)
%\apname{д.ф.-м.~н., доц. Э.~Д.~Кузнецов} % Имя лица, допускающего к защите (зав. кафедрой)
%\control{к.ф.-м.~н., асс. Е.~А.~Аввакумова} %Нормоконтролер
%\title{Выпускная квалификационная\\ работа бакалавра} % Тип
%\author{Иванова Ивана Ивановича} % ФИО автора в родительном падеже
%\sa{инженер-исследователь~АО~УрФУ С.~Ю.~Парфёнов} % Научный руководитель 
%\sasecond{к.ф.-м.~н., зав.~отделом физики Солнца А.~М.~Соболев} %Второй научный руководитель

%%%%%% Титульный лист дипломной работы  %%%%%%
%\spec{03.05.01} %Шифр специальности
%\speciality{Астрономия} % Специальность
%\apname{д.ф.-м.~н., доц. Э.~Д.~Кузнецов} % Имя лица, допускающего к защите (зав. кафедрой или директор департамента)
%\control{к.ф.-м.~н., асс. Е.~А.~Аввакумова} %Нормоконтролер
%\title{Дипломная работа} % Тип
%\author{Иванова Ивана Ивановича} % ФИО автора в родительном падеже
%\sa{инженер-исследователь~АО~УрФУ С.~Ю.~Парфёнов} % Научный руководитель 
%\sasecond{к.ф.-м.~н., зав.~отделом физики Солнца А.~М.~Соболев} %Второй научный руководитель

%%%%%% Титульный лист магистерской диссертации  %%%%%%
%\spec{03.05.01} %Шифр специальности или направления
%\masterprog{Астрофизика} % Название магистерской программы
%\apname{д.ф.-м.~н., доц. Э.~Д.~Кузнецов} % Имя лица, допускающего к защите (зав. кафедрой или директор департамента)
%\control{к.ф.-м.~н., асс. Е.~А.~Аввакумова} %Нормоконтролер
%\title{Магистерская диссертация} % Тип
%\author{Иванова Ивана Ивановича} % ФИО автора в родительном падеже
%\sa{инженер-исследователь~АО~УрФУ С.~Ю.~Парфёнов} % Научный руководитель 
%\sasecond{к.ф.-м.~н., зав.~отделом физики Солнца А.~М.~Соболев} %Второй научный руководитель
 %В этом файле лежат все команды для титульного листа

    %A
\def\apj{Astrophys.~J}
\def\apjl{Astrophys.~J.,~Lett}
\def\apjs{Astrophys.~J.,~Suppl.~Ser}
\def\ap{Astrophysics}
\def\an{Astron.~Nachr}
\def\actaa{Acta Astron}
\def\aap{Astron.~Astrophys}
\def\aaps{Astron.~and Astrophys.,~Suppl.~Ser}
\def\aatr{Astron.~Astrophys.~Trans}
\def\aspcs{ASP~Conf.~Ser}
\def\aj{Astron.~J}
\def\araa{Ann.~Rev.~Astron.~Astrophys}
\def\apss{Astrophys.~Space.~Sci}
\def\astl{Astron.~Letters}
\def\aapr{Astron.~Astrophys.~Rev}
\def\acp{Atmosphere~Chem.~Phys}
\def\aph{Astroparticle~Physics}
\def\adspr{Adv.~Space~Res}
\def\arep{Astron.~Rep}
\def\astbu{Astrophys.~Bull}

%B
\def\bicds{Bull.~Inf.~Cent.~de Donn{'e}es Stellaires}
\def\bastic{Bull.~Astron.~Inst.~Czech}
\def\bain{Bull.~Astron.~Inst.~Neth}
\def\bamass{Bull.~Am.~Astron.~Soc}
\def\bott{Bol.~Obs.~Tonantzintla~Tacubaya}
\def\baltas{Baltic~Astron}

%C
\def\cemda{Celest.~Mech.~Dyn.~Astr}
\def\ceab{Cent.~Eur.~Astrophys.~Bull}
\def\cophc{Comput.~Phys.~Commun}
\def\cemec{Celest.~Mech}
\def\cajph{Can.~J.~Phys}
\def\caosp{Contrib.~Astron.~Obs.~Skalnat{\'e}~Pleso}

%E
\def\epsl{Earth~Planet.~Sci.~Lett}
\def\emp{Earth,~Moon~and~Planets}

%G
\def\geoj{Geophys.~J}
\def\gca{Geochim.~Cosmochim.~Acta}

%I
\def\ibvs{Inf.~Bull.~Variable~Stars}
\def\icar{Icarus}

%J
\def\jaavso{J.~Am.~Assoc.~Variable~Star~Obs}
\def\japa{J.~Astrophys.~Astron}
\def\jcp{J.~Chem.~Phys}
\def\jgr{J.~Geophys.~Res}
\def\jcoph{J.~Comp.~Phys}
\def\jrasc{J.~R.~Astron.~Soc.~Can}
\def\jasa{J.~Acoust.~Soc.~Am}
\def\jfm{J.~Fluid~Mech}
\def\jqsrt{J.~Quant.~Spectrosc.~Radiat.~Transfer}
\def\jbasta{J.~Br.~Astron.~Assoc}

%L
\def\lnm{Lect.~Notes~in~Math}

%M
\def\memsai{Mem.~Soc.~Astron.~Ital}
\def\mnras{Mon.~Not.~R.~Astron.~Soc}
\def\mitag{Mitt.~Astron.~Ges}

%N
\def\na{New~Astron}
\def\nar{New~Astron.~Rev}
\def\nat{Nature}
\def\nimpa{Nucl.~Instrum.~Methods~Phys.~Res.,~Sect.~A}

%O
\def\oejv{Open~Eur.~J.~Variable~Stars}

%P
\def\prl{Phys.~Rev.~Lett}
\def\phrva{Phys.~Rev.~A}
\def\pf{Phys.~Fluids}
\def\pasp{Publ.~Astron.~Soc.~Pac}
\def\phpl{Physics~of~Plasmas}
\def\pasj{Publ.~Astron.~Soc.~Jpn}
\def\pasau{Proc.~Astron.~Soc.~Aust}
\def\puasau{Publ.~Astron.~Soc.~Aust}
\def\phr{Phys.~Rep}
\def\pss{Planet.~Space~Sci}
\def\pz{Perem.~Zvezdy}

%Q
\def\qjras{Q.~J.~R.~Astron.~Soc}

%R
\def\rmxaa{Rev.~Mex.~Astron.~Astrofis}

%S
\def\ssr{Space~Sci.~Rev}
\def\sci{Science}
\def\soph{Sol.~Phys}

%T
\def\obs{The Observatory}

%V
\def\va{Vistas~Astron}
%%%%%%%%%%%%
%Russian
\def\cosiss{Космич.~исслед}
\def\vestcpbu{Вестн.~С.-Петерб.~ун-та}
\def\izvvrad{Изв.~вузов.~Радиофизика}
\def\izvans{Изв.~AH~CCCP}
\def\vestvgu{Вестн.~ВолГУ}
\def\avest{Астрон.~вест}
\def\azh{Астрон.~журн}
\def\pazh{Письма~в~Астрон.~журн}

%%%%%%%%%%%%
\def\thmc{Тез.~международ.~конф}
\def\mmc{Материалы~международ.~конф}
\def\mvrc{Материалы~всерос.~конф}
\def\cntc{Сб.~науч.~тр.~конф}
\def\tmnpc{Тр.~Международ.~науч.-практ.~конф}
\def\ctc{Сб.~тр.~конф}
\def\mcnctone{Тр.~31-й~Международ.~студ.~науч.~конф., Екатеринбург, 28 янв. --- 1 февр. 2002~г}
\def\mcnctto{Тр.~43-й~Международ.~студ.~науч.~конф., Екатеринбург, 3 --- 7 февр. 2014~г}
\def\mcncttr{Тр.~44-й~Международ.~студ.~науч.~конф., Екатеринбург, 2 --- 6 февр. 2015~г}
\def\mcnctfo{Тр.~34-й~Международ.~студ.~науч.~конф., Екатеринбург, 31 янв. --- 4 февр. 2005~г}
\def\mcnctfi{Тр.~35-й~Международ.~студ.~науч.~конф., Екатеринбург, 30 янв. --- 3 февр. 2006~г}
\def\mcnctsi{Тр.~36-й~Международ.~студ.~науч.~конф., Екатеринбург, 29 янв. --- 2 февр. 2007~г}
\def\mcnctse{Тр.~37-й~Международ.~студ.~науч.~конф., Екатеринбург, 28 янв. --- 1 февр. 2008~г}
\def\mcncfo{Тр.~40-й~Международ.~студ.~науч.~конф., Екатеринбург, 31 янв. --- 4 февр. 2011~г}
\def\tmc{Тр.~Международ.~конф}
\def\tmac{Тр.~Международ.~астрон.~конф}
\def\tc{Тр.~конф}
\def\thc{Тез.~конф}

\def\IAUsymp{Proc.~IAU~Symp.}
\def\IAUcoll{Proc.~IAU~Colloquia}
\def\prconf{Proc.~conf.}
\def\aspproc{Proc.~ASP~conf.~ser.}
\def\aipproc{Proc.~AIP~conf.~ser.}
\def\princonf{Proc.~int.~conf.}
 %В этом файле команды для названий журналов
    \maketitle

    \annotate\label{annontate}
%\setcounter{fulltabcnt}{\totvalue{longtabcnt}+\totvalue{footable}}
%\lastpagenumber
%Выпускная квалификационная работа бакалавра \totalnumberofpages{}~с., X ч., X рис., X табл., X прил., XX источников.

%\setcounter{fulltabcnt}{\totvalue{longtabcnt}+\totvalue{footable}}

%Работа содержит \total{page}~страниц, \total{foofigure}~рисунков, \thefulltabcnt~таблицы, \total{citnum}~источников цитирования.
Работа содержит Х страниц, Х рисунков, Х таблицы, Х источников цитирования.

КЛЮЧЕВЫЕ СЛОВА.

Реферат включает следующие аспекты содержания исходного документа:
\begin{asparaitem}
    \item предмет, тему, цель работы;
    \item метод или методологию проведения работы;
    \item результаты работы;
    \item область применения результатов;
    \item выводы;
    \item дополнительную информацию.
\end{asparaitem}

Реферат выполняется в соответствии с ГОСТ~7.9-95. % раздел "РЕФЕРАТ"
% ----------------------------------------------------------------
    \newpage
    \tableofcontents
    \defs
\begin{tabular}{p{2.5cm}p{10cm}}
    АО~УрФУ      & Aстрономическая обсерватория Уральского Федерального Университета;      \\
    TESS         & Transiting Exoplanet Survey Satellite;      \\
    TTF          & TESS Transit Finder;\\
    ExoFOP       & Exoplanet Follow-up Observing Program;\\
    TOI          & TESS Object of Interest;\\
    TEV          & TESS Exoplanet Vetter.

\end{tabular}
 % раздел "ОБОЗНАЧЕНИЯ И СОКРАЩЕНИЯ"
% ----------------------------------------------------------------
    \intro\label{ch:intro}

Сегодня одной из наиболее бурно развивающейся областью астрономии являются экзопланетные исследования.
Множество команд по всему миру ищут и исследуют планеты у других звёзд, из которых
самыми интересными для науки и для общественности являются землеподобные планеты,
расположенные на достаточном расстоянии от родительской звезды, чтобы на планете существовала жидкая вода.
Исследования экзопланетных систем очень важно для нашего понимания формирования и эволюции планет.


Основной метод изучения экзопланет --- это метод транзитной фотометрии.
Он основан на фиксации падения блеска во время прохождения экзопланеты по диску звезды на луче нашего зрения.
Легко догадаться, что такое падение блеска пропорционально отношению площадей экзопланеты и родительской звезды,
таким образом мы можем измерить размер транзитной планеты, не видя её саму.
Знание размеров экзопланеты позволяет оценивать её плотность, а сопутствующие точные спектроскопические
измерения во время транзита и вне его дают нам информацию о составе атмосферы планеты.
В некоторых случаях может быть зарегистрирован свет от самой экзопланеты, тогда, если её размер известен,
можно делать выводы об эффективной температуре.


В данной работе исследуется экзопланетная система TIC229510866 на основе наблюдений с телескопа МАСТЕР-Урал и
архива ExoFOP-TESS при помощи моделирования кривой блеска.
Модель кривой блеска отражает наши физические представления о транзите, следовательно,
необходимо использовать различные модели для описания разных систем.
В работе будут использоваться 4 модели, из которых мы выберем ту, что лучшим образом описывает наблюдательные данные.
На основе этой модели мы сделаем вывод о некоторых параметрах системы, а именно:
об отношении радиусов экзопланеты и родительской звезды, расстоянии между ними,
эксцентриситет, наклон и аргумент перицентра системы.
 % раздел "ВВЕДЕНИЕ"

%\main % печатает заголовок "ОСНОВНАЯ ЧАСТЬ", не включается в курсовую работу
    \newpage
    \input{review} % раздел "Обзор литературы"
%\chapter{Постановка задачи работы}
Целью данной работы является исследование переменности межзвездных молекулярных линий поглощения СН и СН+ в период с 1997 по 2015 год по оптическим спектрам высокого разрешения для звезды AE~Aur. Для этого необходимо решить следующие задачи:
\begin{asparaitem}
\item обработка спектральных данных, полученных на эшелле-спектрографе высокого разрешения, установленного на 1.2-метровом телескопе АО~УрФУ;
\item анализ спектральных данных, полученных на различных инструментах, в том числе 1.2-метровом телескопе АО~УрФУ;
\item оценка эквивалентных ширин линий;
\item оценка лучевой концентрации вещества.
\end{asparaitem} % раздел "Постановка задачи работы", не включается в курсоваую работу

%    \chapter{Способы и методы теоретических и аналитических исследований}\label{ch:3}
%\chapter{Методика эксперимента}
%\chapter{Методика проведения наблюдений и измерений}
%\chapter{Способы и методы решения задачи}


\section{Наблюдательные данные}\label{sec:survey-data}
В данной работе для исследования межзвездных молекулярных линий поглощения использовались спектральные данные высокого разрешения, полученные на различных телескопах за период с 1997 по 2015 год. В общей сложности было обработано 33 спектра, из них:
\begin{asparaitem}
    \item 6 спектров, полученных на спектрографе BOES, установленном на 1.8-метровом рефлекторе в BOAO (Южная Корея);
    \item 21 спектр -- на спектрографе МАЭСТРО, установленном на телескопе Цейсс-2000 в Терскольской обсерватории (Россия);
    \item 1 спектр -- на спектрографе MIKE, установленном на 6.5-метровом Магеллановом телескопе в обсерватории Las Campanas (Чили);
    \item 1 спектр -- на спектрографе UVES, установленном на одном из 8.2-метровых телескопов в Европейской Южной Обсерватории (Чили);
    \item 1 спектр -- на спектрографе HARPS-N, установленном на 3.58-метровом TNG в FGG -- INAF (Испания);
    \item 3 спектра -- на спектрографе UFES, установленном на 1.2-метровом телескопе в АО УрФУ (Россия).
\end{asparaitem}


\section{Методики обработки спектральных данных}\label{sec:method-spec}
Первичная обработка спектральных данных, полученных на спектрографе UFES в АО УрФУ, выполнялась в программном пакете DECH, созданном сотрудником САО РАН Г.~А~Галазутдиновым~\cite{Galazutdinov1992} и включающим в себя программы Dech~95 (spectra imaging) и Dech~20t (spectra processing).
Dech~95 предназначен для просмотра изображения эшелле-спектрограммы и для работы с этим изображением, включающей в себя следующие этапы:
\begin{asparaitem}
    \item медианное усреднение всех изображений (объекта, торий-аргоновой лампы, снимка шума считывания, плоского поля), что позволяет повысить отношение сигнал к шуму и получить изображение, свободное от следов космических частиц;
    \item вычитание среднего кадра снимка шума считывания из всех изображений;
    \item построение маски -- таблицы с координатами положения спектральных порядков.
    Маска строится вручную путем расстановки реперных точек в центре спектральных порядков.
    Достаточно расставить точки на первых двух порядках;
    \item экстракция порядков -- извлечение одномерного спектра из изображения.
\end{asparaitem}

Дальнейшая обработка спектральных данных проводится с помощью программы Dech~20t, предназначенной для работы с экстрагированным спектром в отдельных порядках. Она позволяет:
\begin{asparaitem}
    \item удалять дефекты в спектре, например, плохие пиксели;
    \item сглаживать спектры;
    \item делить их на другие спектры;
    \item строить дисперсионную кривую для одного порядка и всего спектра. Подробное описание процесса построения дисперсионной кривой представлено в подразделе~\ref{subsec:dcm};
    \item измерять лучевые скорости;
    \item строить уровень непрерывного спектра (континуум). Более подробно построение уровня непрерывного спектра рассмотрено в подразделе~\ref{subsec:contin};
    \item измерять эквивалентные ширины линий несколькими способами (см. подраз-\\дел~\ref{subsec:contin});
    \item измерять лучевые концентрации вещества (см. раздел~\ref{sec:n}).
\end{asparaitem}

\subsection{Построение дисперсионной кривой}\label{subsec:dcm}
Построение дисперсионной кривой необходимо для перехода от шкалы в величинах пикселей к шкале длин волн.
Для этого нужно отождествить линии лабораторной (Th-Ar) лампы с линиями атласа для данного инструмента.

Для отождествленной линии ставится маркер и отмечается отсчет по длине волны.
Местоположение маркера определяется по максимальному совпадению профиля линии и его зеркального отражения.
При этом внутри порядка распределение этих маркеров должно идти по возможности максимально равномерно.
По всем маркерам внутри порядка строится средняя дисперсионная кривая.
Положения реперов аппроксимируются полиномом, степень которого задается вручную и не должна превышать значения (n-1), где n~--~количество расставленных маркеров.
Точность построения дисперсионной кривой характеризуется среднеквадратичным отклонением положения маркеров от этой кривой.
Результаты построения дисперсионной кривой для одного порядка представлены на рисунке~\ref{fig:dcm}.
В верхней части рисунка показаны реперы и форма дисперсионной кривой, в нижней~--~отклонения реперных точек от средней дисперсионной кривой, а в таблице представлена информация о каждом репере.
\begin{figure}[h]
    \centering
    \includegraphics[scale=0.65]{figures/dcm.eps}
    \caption{Результаты построения дисперсионной кривой для 55 спектрального порядка}
    \label{fig:dcm}
\end{figure}
Так как изображения звезды и лампы получены на том же приборе и при экстракции спектров используется одна и та же маска, то длина порядков и дисперсионная кривая для изображений звезды и лампы -- одинаковы.

Также программный пакет Dech~20t позволяет строить глобальную дисперсионную кривую для всех спектральных порядков.
Глобальная дисперсионная кривая может быть представлена функцией\eqref{eq:disp}), которая аналитически связывает реперы внутри одного спектрального порядка и сами порядки между собой.

\begin{equation}
    \lambda (x,m) = \sum_{i=0}^P \sum_{j=0}^O a_{ij} x^i m^j, \label{eq:disp}
\end{equation}
где $\lambda$~--~длина волны; \\ $x$~--~координата репера в пикселях; \\ $m$~--~номер порядка; \\ $a_{ij}$~--~коэффициент полинома (определяется методом наименьших квадратов); \\ $P$~--~степень полинома по $x$; \\ $O$~--~степень полинома по $m$.

Глобальная дисперсионная кривая обеспечивает надежную шкалу длин волн, даже в тех порядках, в которых не определены реперные точки.
Качество ее построения напрямую зависит от количества и точности расстановки реперов, а также от степени глобального полинома по пикселям (ByX) и по порядкам (ByM).
Для определения оптимального значения степеней полинома по пикселям и по порядку используется метод простых итераций, встроенный в программный пакет Dech~20t.

\subsection{Методика построения уровня непрерывного спектра и оценки~W в программном пакете Dech~20t}\label{subsec:contin}

Процесс построения уровня непрерывного спектра в программном пакете Dech~20t заключается в ручном расставлении реперных точек (не более 30), приблизительно на середине шумовой дорожки спектра, в местах свободных от линий излучения или поглощения. После чего реперные точки приближаются сплайном, который также должен проходить посередине шума и не искривляться над линией. Данный этап в обработке спектров является наиболее важным и трудоемким, так как ошибки в построении уровня непрерывного спектра напрямую влияют на вид профиля линии, а это, в свою очередь, непосредственно сказывается на точности определения эквивалентной ширины линии и лучевой концентрации вещества.
Нормализованный спектр используется для измерения W линий.
В программном пакете Dech~20t имеется три основных способа выполнения данных измерений:

\begin{asparaitem}
    \item прямое интегрирование.
    Данный способ является наиболее оптимальным и используется по умолчанию.
    Для этого необходимо указать левую и правую границы измеряемой линии, а также определить значение отношения сигнала к шуму, что позволяет оценить ошибки измерений.
    Значение W вычисляется
    по всем точкам профиля линии как:
    \begin{equation}
        W = \int(1-\frac{F_\lambda}{F_0})d\lambda \label{W},
    \end{equation}
    где $F_\lambda$ -- интенсивность точки профиля линии; \\ $F_0$ -- интенсивность континуума; \\ $\lambda$ -- длина волны.
    \item Аппроксимация гауссианой.
    Профиль линии в данном способе приближается вручную с помощью гауссианы, а вычисление значения W производится по формуле~\eqref{W}.
    \item Построение профиля линии вручную.
    В этом способе необходимо определить границы измеряемой линии, после чего следует расставить реперные точки (аналогично методу построения континуума), по которым будет построен профиль линии и измерено значение W по формуле~(\ref{W}).
\end{asparaitem}


\section{Автоматическое приближение уровня непрерывного спектра и оценка W}\label{sec:auto-w}
До сих пор остается нерешенной проблема точности построения уровня непрерывного спектра, так как большую роль в этом играет субъективный человеческий фактор. Как уже упоминалось ранее, неточности в построении уровня непрерывного спектра могут привести к большим ошибкам в оценках эквивалентных ширин линий и лучевой концентрации вещества.

\begin{figure}[h]
    \centering
    \includegraphics[scale=0.75]{figures/EW2.eps}
    \caption{Метод автоматического построения уровня непрерывного спектра на примере линии СН~$\lambda4300$. Синим нарисованы участки спектра, свободные от линии. Зеленым~--~профиль линии. Красным~--~уровень непрерывного спектра}
    \label{fig:continuum}
\end{figure}

Для возможного решения данной проблемы нами был разработан и протестирован метод автоматического построения уровня непрерывного спектра и измерения W, реализованный на языке программирования Python. Для построения континуума и определения W необходимо выбрать три окна. Два из трех окон представляют собой «чистый» континуум -- участки континуума по обе стороны от измеряемой линии свободные от каких-либо линий. В третьем окне определяются границы и профиль самой линии поглощения. Для единообразности измерений W за различные даты границы каждого окна брались фиксированной ширины.

Построение континуума основывается на полиномиальном приближении точек шумовой дорожки, степень полинома можно варьировать от 1 до 3 в зависимости от сложности профиля линии и количества точек в выбранных окнах.
На рисунке~\ref{fig:continuum} представлен пример построения уровня непрерывного спектра для линии СН~$\lambda4300$.

Данная методика позволяет значительно снизить человеческий фактор и тем самым улучшить точность полученных оценок W. Но полностью исключить человеческий фактор с помощью данной методики невозможно, так как необходимо контролировать процесс построения континуума, особенно в области над линией. В случае плохих спектров влияние шума может сильно искривлять уровень непрерывного спектра над линией (см. рисунок~\ref{fig:badcontinuum}). В таком случае, для улучшения точности оценок W, построение континуума выполнялось вручную в программном пакете Dech~20t, для дальнейшего использования уже нормализованного спектра в программе для автоматического построения континуума, в целях его уточнения.
\begin{figure}[h]
    \centering
    \includegraphics[scale=0.75]{figures/badcontinuum.eps}
    \caption{Метод автоматического построения уровня непрерывного спектра на примере линии СН+~$\lambda4232$. Синим нарисованы участки спектра, свободные от линии. Зеленым~--~профиль линии. Красным~--~уровень непрерывного спектра}
    \label{fig:badcontinuum}
\end{figure}

После нормализации спектра, используя формулу~\eqref{W}, вычислялось значение W линии.
Использование данного метода позволяет намного быстрее и точнее учитывать ошибки проведения уровня непрерывного спектра и их вклад в ошибки измерения W, вычисление которых представлено формулами~\eqref{eps}~\cite{Rollinde2003}.
\begin{equation}
    \begin{split}
        \epsilon_r &= \sigma_F * \delta\lambda * sqrt{N}, \\
        \epsilon_a &= \Delta\lambda_\text{line} * |1-F|, \\ \label{eps}
        \epsilon &= \sqrt{\epsilon_r^2 + \epsilon_a^2},
    \end{split}
\end{equation}
где $\epsilon_r$ -- ошибка, связанная с шумом; \\ $\epsilon_a$ -- ошибка, связанная с неопределенностью в уровне континуума; \\ $ \sigma_F$ -- среднеквадратическое отклонение нормализованного потока в континууме; \\ $\delta\lambda$ -- размер пикселя; \\ $N$ -- количество пикселей под линией; \\ $\Delta\lambda_\text{line}$ -- абсолютная ширина линии; \\ $F$ -- среднее значение нормализованного потока в континууме; \\ $\epsilon$ -- ошибка измерения W.

В данной работе использовались как автоматический, так и ручной методы построения уровня непрерывного спектра и измерений W. В таблице~\ref{tab:tab1} приведено сравнение результатов измерений, полученных с использованием каждого из методов.
В этой таблице $W^a$~--~значение, полученное с использованием метода автоматического построения уровня непрерывного спектра, $W$~--~значение, полученное с помощью программы Dech~20t, $dW$~--~ошибки измерений.
Из данной таблицы видно, что существенного различия в оценках W каждым из методов не наблюдается, то есть значения W для одной и той же линии совпадают в пределах $(1-3)\sigma$.
При этом точность измерения у метода автоматического построения уровня непрерывного спектра и оценки W значительно выше.
Оценки W и N, представленные далее по тексту, получены с помощью метода автоматического приближения континуума.
\begin{table} [h]
    \caption{Сравнение методов построения континуума и измерения эквивалентной ширины для линии СН~$\lambda4300$. }
    \label{tab:tab1}
    \begin{center}
        \begin{tabular}{|c|c|c|c|c|c|c|}
            \hline
            Эпоха наблюдения & Спектрограф & Разрешение & $ W$, м\AA & $ dW$, м\AA & $ W^a$, м\AA & $ dW^a$, м\AA \\\hline
            2453040,5        & BOES        & 90000      & 53,24      & 0,53        & 52,65        & 0,36          \\\hline
            2453479,5        & МАЭСТРО     & 120000     & 52,82      & 1,87        & 53,97        & 0,75          \\\hline
            2454500,5        & BOES        & 90000      & 56,43      & 0,49        & 55,25        & 0,46          \\\hline
            2455554,5        & МАЭСТРО     & 40000      & 49,48      & 1,25        & 51,34        & 0,88          \\\hline
            2455924,5        & МАЭСТРО     & 40000      & 48,08      & 1,71        & 51,13        & 1,81          \\\hline
            2455938,5        & MIKE        & 40000      & 52.41      & 0,97        & 50,80        & 0,37          \\\hline
            2456611,5        & HARPS--N    & 115000     & 53,02      & 1,36        & 51,38        & 0,74          \\\hline
        \end{tabular}
    \end{center}
\end{table}

Так же немаловажно отметить, что в ходе измерений могут возникнуть сложности в определении границ линий, так как они могут быть блендированы звездными линиями поглощения или излучения. В качестве примера такого случая можно рассмотреть линию СН+~$\lambda4232$, которая блендирована линией звездного NeII (см. рисунок~\ref{fig:chne}).
\begin{figure}[h]
    \centering
    \includegraphics[scale=0.5]{figures/CHNe.eps}
    \caption{Межзвездная молекулярная линия СН+~$\lambda4232$ и звездная линия NeII. Сплошной линией изображен наблюдаемый спектр, пунктирной~--~теоретический спектр звезды}
    \label{fig:chne}
\end{figure}

Для исключения влияния звездной линии, было использовано несколько способов:
\begin{asparaitem}
    \item cглаживание вручную.
    Производилось непосредственно в программном пакете Dech~20t при помощи расставления реперных точек уровня непрерывного спектра.
    Также Dech~20t позволяет разбивать профиль линии на несколько компонент гауссианами, однако мы не использовали данный метод.
    Следует отметить, что подобные методы могут привести к большим ошибкам в точности измерений, так как профиль звездной линии может быть весьма сложен.\label{method}
    \item расчет W линии NeII на основе теоретических спектров.
    Ранее нами были определены основные физические параметры для данного объекта на основе моделирования звездных атмосфер.
    Уже имеющиеся данные об AE~Aur были использованы для определения обилия элемента в атмосфере и получения теоретического спектра наилучшим образом соответствующего наблюдаемому спектру.
    Затем, с помощью методов, описанных выше, было определено значение W звездной линии поглощения в теоретическом спектре, которое в дальнейшем вычиталось из значения W, полученного для межзвездной и звездной линий поглощения вместе.
\end{asparaitem}

На практике было выявлено, что большого различия между результатами, полученными обоими способами, не наблюдается (см.
таблицу~\ref{tab:tab2}).
Однако использование метода сглаживания вручную, как уже упоминалось ранее, может привести к большому разбросу значений W линии, из-за неточности расстановки реперных точек при построении уровня непрерывного спектра вручную.
\begin{table}[h]
    \caption{Сравнение методов исключения звездных линий поглощения.}
    \label{tab:tab2}
    \begin{center}
        \begin{tabular}{|c|c|c|c|c|c|c|}
            \hline
            Эпоха наблюдения & Спектрограф & Разрешение & $ W$, м\AA & $ dW$, м\AA & $ W^a$, м\AA & $ dW^a$, м\AA \\\hline
            2453040,5        & BOES        & 90000      & 41,83      & 1,24        & 42,79        & 0,59          \\\hline
            2453479,5        & МАЭСТРО     & 120000     & 43,67      & 0,92        & 44,34        & 1,28          \\\hline
            2454500,5        & BOES        & 90000      & 47,04      & 0,73        & 45,78        & 0,63          \\\hline
            2455554,5        & МАЭСТРО     & 40000      & 45,53      & 0,44        & 46,34        & 1,65          \\\hline
            2455924,5        & МАЭСТРО     & 40000      & 39,00      & 2,37        & 37,13        & 0,83          \\\hline
            2455938,5        & MIKE        & 40000      & 37,06      & 0,41        & 37,72        & 0,26          \\\hline
            2456611,5        & HARPS--N    & 115000     & 35,28      & 1,02        & 33,84        & 0,86          \\\hline
        \end{tabular}
    \end{center}
\end{table}


\section{Оценка лучевой концентрации вещества}\label{sec:n}
Оценки лучевой концентрации вещества производятся в рассмотрении оптически тонкого случая, то есть оптическая толщина линий не превышает единицу. Авторами работы \cite{Boisse2009} было показано, что приближение малой оптической толщины применимо для исследуемых линий. В частности об этом свидетельствует то, что кривая роста для этих линий может быть аппроксимирована простой зависимостью \cite{Boisse2009}:
\begin{equation}
    \begin{split}
        N(CH) &= 2,42*10^{11}*W^{1,48} \\
        N(CH+) &= 2,29*10^{11}*W^{1,48}\label{n}
    \end{split}
\end{equation}

Для перехода от величины W к лучевой концентрации нами использовалась зависимость~\eqref{n}.
В таблице~\ref{tab:tabn} представлены некоторые значения оценки лучевой концентрации.
Оценки ошибок N были получены с помощью программного пакета Dech~20t.
\begin{table}[h]
    \caption{Оценки лучевой концентрации вещества для линии СН~$\lambda4300$.}
    \label{tab:tabn}
    \begin{center}
        \begin{tabular}{|c|c|c|c|c|}
            \hline
            Эпоха наблюдения & Спектрограф & Разрешение & $ N, 10^{13}$ см$^{-2} $ & $ dN, 10^{13}$ см$^{-2} $ \\\hline
            2453040,5        & BOES        & 90000      & 8,54                     & 0,17                      \\\hline
            2453479,5        & МАЭСТРО     & 120000     & 8,86                     & 0,25                      \\\hline
            2454500,5        & BOES        & 90000      & 9,17                     & 0,85                      \\\hline
            2455554,5        & МАЭСТРО     & 40000      & 8,23                     & 0,36                      \\\hline
            2455924,5        & МАЭСТРО     & 40000      & 8,18                     & 0,44                      \\\hline
            2455938,5        & MIKE        & 40000      & 8,10                     & 0,05                      \\\hline
            2456611,5        & HARPS--N    & 115000     & 8,24                     & 0,22                      \\\hline
        \end{tabular}
    \end{center}
\end{table} % раздел "Способы и методы теоретических и аналитических исследований"
    %! Author = Nikita
%! Date = 12.03.2021


\chapter{Моделирование транзита по наблюдениям телескопа МАСТЕР-Урал при помощи различных моделей потемнения к краю}
\label{ch:modeling}


    \section{Моделирование с использованием модели однородного диска}\label{sec:modeling-disc}


    \section{Моделирование с использованием Quadratic model}\label{sec:modeling-quadratic-model}


    \section{Моделирование с использованием QPower2 model}\label{sec:моделирование-с-использованием-qpower2-model}



%    \chapter{Результаты и их обсуждение}
В ходе данной работы было обработано 33 спектра звезды HD~34078, полученных на эшелле-спектрографах высокого разрешения за период времени с 1997 года по 2015 год. Это значительно больший промежуток времени, по сравнению с работой \cite{Boisse2009}, в которой наблюдательные данные представлены лишь до 2007 года.

По результатам, полученным в ходе выполнения данной работы, можно сделать вывод о том, что нами были обнаружены вариации параметров линий поглощения СН и СН+, так как разброс их значений за исследованный период времени довольно велик. Проанализировав полученные данные, можно говорить о видимом уменьшении W линии СН+ и, соответственно, лучевой концентрации вещества, примерно на 40~\% за последние 18 лет (см. рисунки~\ref{pic:WCH} и~\ref{pic:NCH}). Причем наиболее интенсивный спад наблюдается примерно за последние четыре года. В то же время изменение W линии СН за исследованный промежуток времени происходило в пределах точности измерений (см. рисунки~\ref{pic:WCH} и~\ref{pic:NCH}). 
 
\begin{figure}[h]
\centering
\includegraphics[scale=0.5]{figures/WCH.eps}
\caption{Зависимость W линий СН~$\lambda4300$ и CH+~$\lambda4232$ от времени. Красным отмечены данные, для которых измерения W затруднены из-за относительно низкого отношения сигнала к шуму и блендирования линий
}
\label{pic:WCH}
\end{figure} 
\begin{figure}[h]
\centering
\includegraphics[scale=0.4]{figures/N.eps}
\caption{Зависимость N для линий СН~$\lambda4300$ и CH+~$\lambda4232$ от времени. Красным отмечены данные, для которых измерения W, а следовательно и N, затруднены из-за относительно низкого отношения сигнала к шуму и блендирования линий
}
\label{pic:NCH}
\end{figure} 

В работе \cite{Boisse2009} также были отмечены вариации W исследуемых линий поглощения, однако сама зависимость от времени имеет другой вид, по сравнению с зависимостью, полученной нами (см. приложение~\ref{Boisse}). При этом в работе \cite{Boisse2009} также использовались некоторые спектры, применявшиеся для оценок W и N в данной работе. Невозможно сказать точно, с чем связаны подобные различия, однако это не влияет на вывод о том, что соотношение W линий CН и СН+ заметно изменяется за  последние четыре года.

Так же наглядно можно подтвердить полученные результаты сравнивая профили линий поглощения за разные даты, полученные на спектрографах с одинаковым спектральным разрешением. На рисунке~\ref{pic:profiles} показано сравнение профилей линий СН+ и CH за разные даты. Как можно видеть, интенсивность линий СН+ значительно уменьшилась за период времени с 2005 год по 2012 год, в то время как профили линий СН остаются неизменными. 
\begin{figure}[h]
\centering
\includegraphics[scale=0.65]{figures/profiles.eps}
\caption{Сравнение профилей линий СН и СН+. Рисунок взят из работы~\cite{Jacek}}
\label{pic:profiles}
\end{figure} 

Наиболее интересные результаты были получены во время последних наблюдений на спектрографе UFES 22 января и 4  марта 2015 года. Наблюдается значительное уменьшение эквивалентной ширины линии СН+~$\lambda$4232, за эти даты примерно на 30~\%, в то время как изменение в значении эквивалентной ширины линии СН~$\lambda$4300 менее интенсивно. Причина данных изменений может быть связана с вариациями физических условий на небольших масштабах в окрестности звезды AE~Aur. В то время как вариации физических условий могут быть связаны с взаимодействием ударной волны и излучения, идущих от звезды, с достаточно плотными сгустками газа в ее окрестности. Это согласуется с выводами полученными в работе~\cite{Gratier2014}. % раздел "Результаты и их обсуждение"
    %! Author = Nikita
%! Date = 12.03.2021

\section{Сравнение результатов между собой и с данными каталога ExoFOP-TESS}\label{sec:analisis}

%    \conclusion\label{ch:concl}
В ходе данной работы была выполнена обработка 33 оптических спектров высокого разрешения звезды AE~Aur. 
Был разработан и протестирован метод автоматического построения уровня непрерывного спектра, который позволяет с высокой точность проводить континуум и измерять эквивалентные ширины линий. Высокая точность в данном методе достигается из-за того, что при выполнении измерений в значительной мере снижается человеческий фактор. 
С использованием данного метода были произведены оценки эквивалентных ширин межзвездных молекулярных линий поглощения СН и СН+ и лучевой концентрации вещества. 
Так же был проведен анализ различных методов оценок измеряемых величин, результаты которых были представлены выше.

В результате выполнения данной работы можно сказать о том, что были обнаружены вариации физических условий в межзвездной среде в окрестности звезды HD~34078 на небольших пространственных масштабах. Наиболее значительные изменения наблюдаются в линиях СН+, в частности были обнаружены заметные вариации на небольших временных масштабах порядка месяца. В то время как изменения в линии СН незначительны и происходят в пределах точности измерений.

Стоит отметить, что наблюдательных данных, представленных в данной работе недостаточно для детального описания физических процессов, приводящих к полученным результатам. Однако наблюдения за длительный промежуток времени могут быть довольно полезными для разработки сценария, наилучшим образом согласующегося с наблюдениями.
 % раздел "ЗАКЛЮЧЕНИЕ"
    \input{result.tex}

    \bibliographystyle{Kourovkastyle} %Стилевой файл для списка литературы в BibTeX
    \bibliography{References} % подключение списка литературы, сделанного в формате .bib

% ----------------------------------------------------------------
    \appendix % Раздел с Приложениями

    \chapter[Название]{(справочное/обязательное)\\[-2mm]Название}\label{ch:sss}
\renewcommand{\baselinestretch}{1.}
\newpage
%\chapter[Оценки эквивалентных ширин линий, полученные ранее]{(справочное)}\label{ch:boisse}
%
%\begin{figure}[h]
%\centering
%\includegraphics[scale=0.65]{figures/Boisse.eps}
%\caption{Результаты оценок эквивалентных ширин линий, приведенные в работе~\cite{Boisse2009}}\label{fig:figure}
%\end{figure} % Файл с текстом приложения

\end{document}
