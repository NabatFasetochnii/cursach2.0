\begin{center}
    \textsc{
    ИЗУЧЕНИЕ ЗАКОНОМЕРНОСТЕЙ РАДИОАКТИВНОГО РАСПАДА\\
    }
    Лабораторная работа №8
\end{center}
\begin{flushleft}
    Выполнил: Чазов Никита Андреевич\\
    Выполнена 11.05.2021\\
    Цель работы:
    \begin{enumerate}
        \item Подтверждение случайного, статистического характера процессов радиоактивного распада ядер.
        \item Ознакомление с основными понятиями и методами статистического анализа данных.
    \end{enumerate}

\end{flushleft}

\chapter{Краткая теория}\label{ch:theory}
Многие физические величины по своей природе являются случайными.
В таком случае, чтобы обработать измерения таких величин, нам понадобится найти их статистическое распределение.
В данной работе изучается активность космического фона при помощи распределения Пуассона и двух критериев согласия:
критерий Пирсона и критерий Колмогорова.

Можно получить временную зависимость активности источника из основного закона радиоактивного распада:

%$d^5$
%\[
%    X = X_0 e^{-\lambda t}
%%    \pd{x}{t}=\pd[2]{u}{r^2}
%
%\]



