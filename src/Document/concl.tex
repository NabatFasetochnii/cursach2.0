\conclusion\label{ch:concl}
В ходе данной работы была выполнена обработка 33 оптических спектров высокого разрешения звезды AE~Aur. 
Был разработан и протестирован метод автоматического построения уровня непрерывного спектра, который позволяет с высокой точность проводить континуум и измерять эквивалентные ширины линий. Высокая точность в данном методе достигается из-за того, что при выполнении измерений в значительной мере снижается человеческий фактор. 
С использованием данного метода были произведены оценки эквивалентных ширин межзвездных молекулярных линий поглощения СН и СН+ и лучевой концентрации вещества. 
Так же был проведен анализ различных методов оценок измеряемых величин, результаты которых были представлены выше.

В результате выполнения данной работы можно сказать о том, что были обнаружены вариации физических условий в межзвездной среде в окрестности звезды HD~34078 на небольших пространственных масштабах. Наиболее значительные изменения наблюдаются в линиях СН+, в частности были обнаружены заметные вариации на небольших временных масштабах порядка месяца. В то время как изменения в линии СН незначительны и происходят в пределах точности измерений.

Стоит отметить, что наблюдательных данных, представленных в данной работе недостаточно для детального описания физических процессов, приводящих к полученным результатам. Однако наблюдения за длительный промежуток времени могут быть довольно полезными для разработки сценария, наилучшим образом согласующегося с наблюдениями.
