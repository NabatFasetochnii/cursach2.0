\intro\label{ch:intro}

Сегодня одной из наиболее бурно развивающейся областью астрономии являются экзопланетные исследования.
Множество команд по всему миру ищут и исследуют планеты у других звёзд, из которых
самыми интересными для науки и для общественности являются землеподобные планеты,
расположенные на достаточном расстоянии от родительской звезды, чтобы на планете существовала жидкая вода.
Исследования экзопланетных систем очень важно для нашего понимания формирования и эволюции планет.


Основной метод изучения экзопланет --- это метод транзитной фотометрии.
Он основан на фиксации падения блеска во время прохождения экзопланеты по диску звезды на луче нашего зрения.
Легко догадаться, что такое падение блеска пропорционально отношению площадей экзопланеты и родительской звезды,
таким образом мы можем измерить размер транзитной планеты, не видя её саму.
Знание размеров экзопланеты позволяет оценивать её плотность, а сопутствующие точные спектроскопические
измерения во время транзита и вне его дают нам информацию о составе атмосферы планеты.
В некоторых случаях может быть зарегистрирован свет от самой экзопланеты, тогда, если её размер известен,
можно делать выводы об эффективной температуре.


В данной работе исследуется экзопланетная система TIC229510866 на основе наблюдений с телескопа МАСТЕР-Урал и
архива ExoFOP-TESS при помощи моделирования кривой блеска.
Модель кривой блеска отражает наши физические представления о транзите, следовательно,
необходимо использовать различные модели для описания разных систем.
В работе будут использоваться 4 модели, из которых мы выберем ту, что лучшим образом описывает наблюдательные данные.
На основе этой модели мы сделаем вывод о некоторых параметрах системы, а именно:
об отношении радиусов экзопланеты и родительской звезды, расстоянии между ними,
эксцентриситет, наклон и аргумент перицентра системы.
