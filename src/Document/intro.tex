\intro\label{ch:intro}

Сегодня одной из наиболее бурно развивающейся областью астрономии являются экзопланетные исследования.
Множество команд по всему миру ищут и исследуют планеты у других звёзд, из которых самыми интересными для науки и для общественности
являются землеподобные планеты, расположенные на достаточном расстоянии от родительской
звезды, чтобы на планете существовала жидкая вода.
Исследования экзопланетных
систем очень важно для нашего понимания формирования и эволюции планет.

Основной метод изучения экзопланет — это метод транзитной фотометрии.
Он основан на фиксации падения блеска во время прохождения экзопланеты по диску звезды на луче нашего зрения.
Легко догадаться, что такое падение блеска пропорционально
отношению площадей экзопланеты и родительской звезды, таким образом мы можем
измерить размер отношение радиусов планеты и звезды, не наблюдая их непосредственно.

Если мы знаем спектральный класс звезды, мы можем принять типичный для
этого класса размер звезды и сделать оценку размера планеты.
Сопутствующие точные спектроскопические измерения позволяют сделать оценку массы звезды с точностью
до синуса угла наклона орбиты, найдя этот угол из модели транзита можно более
смело говорить о массе родительской звезды, а как следствие и о массе самой планеты.
После этого уже можно говорить о плотности планеты.
Также, из геометрических
соображений из кривой блеска можно найти среднюю плотность звезды.

В данной работе исследуется экзопланетная система TIC229510866 на основе
наблюдений с телескопа МАСТЕР-Урал и архива ExoFOP-TESS при помощи сопоставления
наблюдений и модели.
Определяются следующие параметры системы: наклонение орбиты
относительно луча зрения, отношение радиуса планеты к радиусу звезды, эксцентриситет,
аргумент перицентра, большая полуось отнесенная к радиусу звезды.

Модель кривой блеска отражает наши физические представления о транзите,
следовательно, необходимо использовать различные модели для описания разных систем.
Обычно модели транзита принимают на вход следующие величины: отношение радиусов
планеты и звезды, коэффициенты потемнения к краю (если модель учитывает этот
эффект), время центра транзита, наклонение орбиты относительно луча зрения, эксцентриситет,
аргумент перицентра, большая полуось отнесенная к радиусу звезды.
В работе будут использоваться 3 модели, из которых мы выберем ту, что лучшим образом описывает
наблюдательные данные.
На основе этой модели мы сделаем вывод о перечисленных параметрах системы.
