\annotate\label{annontate}
%\setcounter{fulltabcnt}{\totvalue{longtabcnt}+\totvalue{footable}}
%\lastpagenumber
%Выпускная квалификационная работа бакалавра \totalnumberofpages{}~с., X ч., X рис., X табл., X прил., XX источников.

%\setcounter{fulltabcnt}{\totvalue{longtabcnt}+\totvalue{footable}}

%Работа содержит \total{page}~страниц, \total{foofigure}~рисунков, \thefulltabcnt~таблицы, \total{citnum}~источников цитирования.
Работа содержит Х страниц, Х рисунков, Х таблицы, Х источников цитирования.

КЛЮЧЕВЫЕ СЛОВА.

Реферат включает следующие аспекты содержания исходного документа:
\begin{asparaitem}
    \item предмет, тему, цель работы;
    \item метод или методологию проведения работы;
    \item результаты работы;
    \item область применения результатов;
    \item выводы;
    \item дополнительную информацию.
\end{asparaitem}

Реферат выполняется в соответствии с ГОСТ~7.9-95.