\subsection{Большая полуось}\label{subsec:semi-major}
Зная орбитальный период планеты и спектральный класс звезды, благодаря третьему закону кеплера, мы можем
оценить большую полуось
\begin{gather}
    \frac{a^3}{P^2}=\frac{G(M_*+M_p)}{4\pi^2},
\end{gather}
где $a$ --- большая полуось орбиты планеты, $P$~---~орбитальный период планеты, $M_*$~---~масса звезды,
$M_{p}$~---~масса планеты.

Пренебрегая массой планеты и оценивая массу звезды из её спектрального класса получаем следующую
зависимость для большой полуоси
\begin{gather}
    a\approx \sqrt[3]{GM_*\left( \frac{P}{2\pi} \right)^2} \label{kepler}
\end{gather}

\subsection{Орбитальная скорость}\label{subsec:orb_veliosity}
Для круговой орбиты орбитальная скорость является постоянной и задается
\begin{gather}
    v=\frac{2\pi a}{P}
\end{gather}
\subsection{Наклонение орбиты, прицельный параметр и время транзита}\label{subsec:mix}
    Транзит имеет 4 ключевые точки: момент касания диска планеты с диском звезды,
    когда планета ещё не затмевает звезду~---~вход в транзит,
    моменты начала и конца плоской части кривой блеска~---~когда планета касается внутренней части края звезды
    и момент касания диска планеты с диском звезды,
    когда планета уже не затмевает звезду~---~выход из транзита.
    Эти позиции проиллюстрированы на рис.~\ref{fig:points}.

    Также, на рис.~\ref{fig:points} показан прицельный параметр b~---~наименьшее расстояние между центром
    звезды и центром планеты в проекции на небо,
    то есть расстояние между центрами звезды и планеты в середине транзита.
    \begin{figure}[h!]
        \begin{center}
            \includegraphics[width=10cm]{3}
            \caption{Положение диска планеты относительно диска звезды в четырех точках контакта.}
            \label{fig:points}
        \end{center}
    \end{figure}

    Прицельный параметр связан с наклонением орбиты, i, следующим соотношением
    \begin{gather}
        b=a \cos i
    \end{gather}
    Однако, в некоторой литературе за прицельный параметр принята безразмерная величина $b=\frac{a \cos i}{R_*}$.
    Геометрия прицельного параметра и наклонение орбиты изображена на рис.~\ref{fig:impact}.
    \begin{figure}[h!]
        \begin{center}
            \includegraphics[width=10cm]{1.png}
            \caption{Геометрия транзита с отличным от нуля прицельным параметром, b, или что тоже самое,
                отличным от 90\textdegree~наклонением.}
            \label{fig:impact}
        \end{center}
    \end{figure}

    Продолжительность транзита сильно зависит от прицельного параметра.
    Так для круговой орбиты продолжительность транзита будет
    \begin{gather}
        T_{dur}=P\frac{\alpha}{2\pi}, \label{dur}
    \end{gather}
    где $\alpha$~---~это длина дуги орбиты от первого касания до последнего.

    На рис.~\ref{fig:time} показан прямоугольный треугольник, который имеет гипотенузу длины $R_* + R_p$ и
    вертикальную сторону, равную прицельному параметру b, и, следовательно, имеет длину $a \cos i$.
    Третья, горизонтальная, сторона треугольника соединяет положения центра диска планеты в середине транзита
    и четвертого контакта.
    По теореме Пифагора длина этой стороны равна

    \begin{gather}
        l=\sqrt{\left(R_*+R_p \right)^2-a^2\cos^2 i}
    \end{gather}

    Эта величина есть лишь проекция на небо пути планеты, но в нашем случае это допустимое приближение.
    \begin{figure}[h!]
        \begin{center}
            \includegraphics[width=10cm]{4.png}
            \caption{К расчёту продолжительности транзита в случае круговой орбиты.}
            \label{fig:time}
        \end{center}
    \end{figure}

    Следует заметить, что из данного изображения так же можно получить предельные значения прицельного параметра,
    при которых транзит будет происходить

    \begin{gather}
        0\leq a \cos i < R_*+R_p\\
        0\leq \cos i < \frac{R_*+R_p}{a}\approx\frac{R_*}{a}
    \end{gather}
    Но при таком предельном значении диск планеты не полностью будет заходить на диск звезды,
    а значит будет особенным образом изменяться площадь скрываемой части диска звезды.
    Полностью планета будет заходить на диск звезды при

    \begin{gather}
        0\leq \cos i < \frac{R_*-R_p}{a}
    \end{gather}

    Угол $\alpha/2$ в уравнении~\ref{dur} можно приближённо считать как $\arcsin \frac{l}{a}$, тогда
    \begin{gather}
        T_{dur}=\frac{P}{\pi}\arcsin\frac{l}{a}=\frac{P}{\pi}\arcsin \left(\frac{\sqrt{\left(R_*+R_p \right)^2-a^2\cos^2 i}}{a} \right)\\
        T_{dur} \approx \frac{P}{\pi}\arcsin \left(\sqrt{\frac{R_*^2}{a^2}-\cos^2 i} \right) \label{inc}
    \end{gather}

    Последняя выкладка справедлива при $a\gg R_*\gg R_p$.
    Если мы знаем спектральный тип звезды, мы можем вывести приблизительные значения $M_*$ и $R_*$, предположив,
    что они типичны для спектрального типа.
    Тогда продолжительность транзита и орбитальный период достаточны для вывода при помощи
    уравнений~\ref{inc}~и~\ref{kepler} приблизительного наклона орбиты.

    Кроме того, как только наклонение орбиты будет оценено таким образом, мы можем использовать время,
    прошедшее между первым и вторым контактом, чтобы получить вторую оценку радиуса планеты.
    При транзитном входе диск планеты перемещается от первого контакта ко второму контакту.

    \begin{gather}
        T_{1-2}=\frac{P}{\pi}\arcsin \left(\frac{\sqrt{\left(R_*-R_p \right)^2-a^2\cos^2 i}}{a}\right)
    \end{gather}
    Время, необходимое для этого, зависит от наклона орбиты, размеров, продолжительности транзита и периода орбиты,
    дают нам первую и последнюю из этих величин, поэтому ранее известное значение для $R_*$,
    а измерение t1-2 подразумевает значение $R_p$.
    Конечно, глубина транзита уже дала нам оценку соотношения радиуса: