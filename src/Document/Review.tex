%! Author = Nikita
%! Date = 11.03.2021


\chapter{Обзор литературы}\label{ch:lit-review}
Мы можем узнать подробности об экзопланетах и их родительских звездах, наблюдая их общий свет,
без необходимости пространственного разрешения планеты и звезды.
В этой главе показано, как наблюдения затмения используются для получения знаний об орбите планеты, массе, радиусе,
температуре и составляющих атмосферы, а также о других деталях, которые в противном случае скрыты.

Затмение --- это перекрытие одного небесного тела другим.
Когда тела имеют кардинально отличающиеся размеры, прохождение меньшего тела
перед большим телом на луче зрения наблюдателя является транзитом,
а прохождение меньшего тела за большим телом~---~вторичным транзитом.


\section{Геометрия транзита и параметры системы}\label{sec:geometr}
\subsection{Координаты планеты}\label{subsec:coord}
Рассмотрим планету радиуса $R_p$ и массы $M_p$, вращающуюся вокруг звезды радиуса $R_*$ массой $M_*$.
Мы выбираем систему координат, центрированную на звезде, плоскость OXY совпадает с плоскостью неба,
ось Z направлена на наблюдателя.
Сонаправим ось OX с линией узлов, поместим нисходящий узел орбиты планеты вдоль положительного направления оси OX,
тогда $\Omega = 180^\circ$.

Расстояние между планетой и звездой задаётся уравнением:
\begin{gather}
    r = \frac{a\left( 1-e^2 \right)}{1+e \cos f},
\end{gather}
где $a$ --- большая полуось орбиты, $f$ --- истинная аномалия, $e$ --- эксцентриситет орбиты.

Выразим декартовы координаты планеты:
\begin{gather}
    X = -r \cos (\omega + f),\\
    Y = -r \sin (\omega + f) \cos i,\\
    Z =  r \sin (\omega + f) \sin i,
\end{gather}
где $i$ --- наклонение орбиты.

Планета находится в перицентре при минимальном значении $\sqrt {X^2+Y^2}$ и
в апоцентре при максимальном значении этого выражения, тогда

\begin{gather}
    \sqrt {X^2+Y^2}=\frac{a(1-e^2)}{1+e \cos f}\sqrt {1-\sin^2 (\omega +f)\sin^2 i},
\end{gather}

Исходя из этого, в зависимости от типа затмения можно сделать два приближения для истинной аномалии
в момент транзита~\cite{Kipping_2008}:

\begin{gather}
    f_1=\frac{\pi}{2}-\omega,\\
    f_2=-\frac{\pi}{2}-\omega,
\end{gather}
где $f_1$ характеризует первичное затмение, а $f_2$ --- вторичное.

\subsection{Большая полуось}\label{subsec:semi-major}
Зная орбитальный период планеты и спектральный класс звезды, благодаря третьему закону кеплера, мы можем
оценить большую полуось:
\begin{gather}
    \frac{a^3}{P^2}=\frac{G(M_*+M_p)}{4\pi^2},
\end{gather}
где $a$ --- большая полуось орбиты планеты, $P$~---~орбитальный период планеты.

Пренебрегая массой планеты и оценивая массу звезды из её спектрального класса получаем следующую
зависимость для большой полуоси:
\begin{gather}
    a\approx \sqrt[3]{GM_*\left( \frac{P}{2\pi} \right)^2}. \label{kepler}
\end{gather}

%\subsection{Орбитальная скорость}\label{subsec:orb_veliosity}
%В простейшем случае круговой орбиты орбитальная скорость является постоянной и задается
%\begin{gather}
%    v=\frac{2\pi a}{P}.
%\end{gather}

\subsection{Наклонение орбиты, прицельный параметр и время транзита}\label{subsec:mix}
Транзит имеет 4 ключевые точки: момент касания диска планеты с диском звезды,
когда планета ещё не затмевает звезду~---~вход в транзит,
моменты начала и конца плоской части кривой блеска~---~когда планета касается внутренней части края звезды
и момент касания диска планеты с диском звезды,
когда планета уже не затмевает звезду~---~выход из транзита.
Эти позиции проиллюстрированы на рис.~\ref{fig:points}.

Также, на рис.~\ref{fig:points} показан прицельный параметр b~---~наименьшее расстояние между центром
звезды и центром планеты в проекции на небо,
то есть расстояние между центрами звезды и планеты в середине транзита.
\begin{figure}[h!]
    \begin{center}
        \includegraphics[width=10cm]{3}
        \caption{Положение диска планеты относительно диска звезды в четырех точках контакта.}
        \label{fig:points}
    \end{center}
\end{figure}

Прицельный параметр связан с наклонением орбиты, i, следующим соотношением
\begin{gather}
    b=a \cos i.
\end{gather}
Однако, в некоторой литературе, как например в~\cite{Kipping_2008} или в~\cite{winn2014transits},
за прицельный параметр принята безразмерная величина $b=\frac{a \cos i}{R_*}$.
В случае эллиптичной орбиты

\begin{gather}
    b_1=a \cos i \left( \frac{1-e^2}{1+e \sin \omega} \right),\\
    b_2=a \cos i \left( \frac{1-e^2}{1-e \sin \omega} \right).
\end{gather}

Геометрия прицельного параметра и наклонение орбиты изображена на рис.~\ref{fig:impact}.
\begin{figure}[h!]
    \begin{center}
        \includegraphics[width=10cm]{1}
        \caption{Геометрия транзита с отличным от нуля прицельным параметром, b, или что тоже самое,
            отличным от 90\textdegree~наклонением.}
        \label{fig:impact}
    \end{center}
\end{figure}

Продолжительность транзита сильно зависит от прицельного параметра.
Так для круговой орбиты продолжительность транзита будет
\begin{gather}
    T_{tot}=P\frac{\alpha}{2\pi}, \label{tot}
\end{gather}
где $\alpha$~---~это длина дуги орбиты от первого касания до последнего.

На рис.~\ref{fig:time} показан прямоугольный треугольник, который имеет гипотенузу длины $R_* + R_p$ и
вертикальную сторону, равную прицельному параметру b, и, следовательно, имеет длину $a \cos i$.
Третья, горизонтальная, сторона треугольника соединяет положения центра диска планеты в середине транзита
и четвертого контакта.
По теореме Пифагора длина этой стороны:

\begin{gather}
    l=\sqrt{\left(R_*+R_p \right)^2-a^2\cos^2 i}.
\end{gather}

Эта величина есть лишь проекция на небо пути планеты, но в нашем случае это допустимое приближение.
\begin{figure}[h!]
    \begin{center}
        \includegraphics[width=10cm]{4}
        \caption{К расчёту продолжительности транзита в случае круговой орбиты.}
        \label{fig:time}
    \end{center}
\end{figure}

Затмения могут быть видны только определённым наблюдателям, которые видят орбиту планеты почти с ребра.
Следует заметить, что из рис.~\ref{fig:time} так же можно получить предельные значения прицельного параметра,
при которых транзит будет происходить

\begin{gather}
    0\leq a \cos i < R_*+R_p,\\
    0\leq \cos i < \frac{R_*+R_p}{a}\approx\frac{R_*}{a}.
\end{gather}
Но при таком предельном значении диск планеты не полностью будет заходить на диск звезды,
а значит будет особенным образом изменяться площадь скрываемой части диска звезды.
Полностью планета будет заходить на диск звезды при

\begin{gather}
    0\leq \cos i < \frac{R_*-R_p}{a}.
\end{gather}

Когда планета вращается вокруг своей звезды, ее тень описывает конус, который выметает полосу на небесной сфере,
Удаленный наблюдатель в пределах полосы тени увидит транзиты.
Угол раскрытия конуса, $\Theta$, удовлетворяет условию $\sin \Theta=( R_* + R_p)/r$,
где r - мгновенное расстояние между звездой и планетой.
Этот конус называется полутень.
Существует также внутренний конус --- область полной тени --- определяемая $\sin \Theta=( R_* - R_p)/r$,
внутри которого планета полностью заходит на диск звезды.

Угол $\alpha/2$ в уравнении~\ref{tot} можно приближённо считать как $\arcsin \frac{l}{a}$, тогда
\begin{gather}
    T_{tot}=\frac{P}{\pi}\arcsin\frac{l}{a}=\frac{P}{\pi}\arcsin \left(\frac{\sqrt{\left(R_*+R_p \right)^2-a^2\cos^2 i}}{a} \right)\\
    T_{tot} \approx \frac{P}{\pi}\arcsin \left(\sqrt{\frac{R_*^2}{a^2}-\cos^2 i} \right) \label{inc}
\end{gather}

Последняя выкладка справедлива при $a\gg R_*\gg R_p$.
Так как дуга орбиты между вторым и третьим касаниями меньше на два радиуса планеты,
то продолжительность плоской части транзита:

\begin{gather}
    T_{full}=\frac{P}{\pi}\arcsin \left(\frac{\sqrt{\left(R_*-R_p \right)^2-a^2\cos^2 i}}{a} \right)
\end{gather}

Приближение для эллиптической орбиты~\cite{winn2014transits}:

\begin{gather}
    T_{tot}=\frac{\sqrt{1-e^2}}{1\pm e \sin \omega }\frac{P}{\pi}\arcsin \left(\frac{\sqrt{\left(R_*+R_p \right)^2-a^2\cos^2 i}}{a} \right),\\
    T_{full}=\frac{\sqrt{1-e^2}}{1\pm e \sin \omega }\frac{P}{\pi}\arcsin \left(\frac{\sqrt{\left(R_*-R_p \right)^2-a^2\cos^2 i}}{a} \right),
\end{gather}
где "$+$"\, соответствует первичному транзиту, а "$-$"\, --- вторичному.

Если мы знаем спектральный тип звезды, мы можем вывести приблизительные значения $M_*$ и $R_*$, предположив,
что они типичны для спектрального типа.
Тогда продолжительность транзита и орбитальный период достаточны для вывода при помощи
уравнений~\ref{inc}~и~\ref{kepler} приблизительного наклона орбиты.

Время входа в транзит, $\tau_{i}$, и выхода из транзита, $\tau_{e}$, равны для круговой орбиты в силу симметрии задачи,
в отличие от эллиптической орбиты, для которой существует оценка различия этих времён~\cite{winn2014transits}:
\begin{gather}
    \frac{\tau_e-\tau_i}{\tau_e+\tau_i}\sim e \cos \omega \left( \frac{R_*}{a} \right)^3 \sqrt{\left( 1-(R_*b)^2 \right)^3}.
\end{gather}
Для близких к своей звезде планет, $R_*/a=0.2$, это выражение будет меньше, чем $10^{-2} e$,
когда же для более далёких планет оно значительно меньше.
Поэтому с хорошей точностью можно принять $\tau_i=\tau_e=\tau$ в общем случае.



%\section{Параметры орбиты}\label{sec:param}
%\subsection{Большая полуось}\label{subsec:semi-major}
Зная орбитальный период планеты и спектральный класс звезды, благодаря третьему закону кеплера, мы можем
оценить большую полуось
\begin{gather}
    \frac{a^3}{P^2}=\frac{G(M_*+M_p)}{4\pi^2},
\end{gather}
где $a$ --- большая полуось орбиты планеты, $P$~---~орбитальный период планеты, $M_*$~---~масса звезды,
$M_{p}$~---~масса планеты.

Пренебрегая массой планеты и оценивая массу звезды из её спектрального класса получаем следующую
зависимость для большой полуоси
\begin{gather}
    a\approx \sqrt[3]{GM_*\left( \frac{P}{2\pi} \right)^2} \label{kepler}
\end{gather}

\subsection{Орбитальная скорость}\label{subsec:orb_veliosity}
Для круговой орбиты орбитальная скорость является постоянной и задается
\begin{gather}
    v=\frac{2\pi a}{P}
\end{gather}
\subsection{Наклонение орбиты, прицельный параметр и время транзита}\label{subsec:mix}
    Транзит имеет 4 ключевые точки: момент касания диска планеты с диском звезды,
    когда планета ещё не затмевает звезду~---~вход в транзит,
    моменты начала и конца плоской части кривой блеска~---~когда планета касается внутренней части края звезды
    и момент касания диска планеты с диском звезды,
    когда планета уже не затмевает звезду~---~выход из транзита.
    Эти позиции проиллюстрированы на рис.~\ref{fig:points}.

    Также, на рис.~\ref{fig:points} показан прицельный параметр b~---~наименьшее расстояние между центром
    звезды и центром планеты в проекции на небо,
    то есть расстояние между центрами звезды и планеты в середине транзита.
    \begin{figure}[h!]
        \begin{center}
            \includegraphics[width=10cm]{3}
            \caption{Положение диска планеты относительно диска звезды в четырех точках контакта.}
            \label{fig:points}
        \end{center}
    \end{figure}

    Прицельный параметр связан с наклонением орбиты, i, следующим соотношением
    \begin{gather}
        b=a \cos i
    \end{gather}
    Однако, в некоторой литературе за прицельный параметр принята безразмерная величина $b=\frac{a \cos i}{R_*}$.
    Геометрия прицельного параметра и наклонение орбиты изображена на рис.~\ref{fig:impact}.
    \begin{figure}[h!]
        \begin{center}
            \includegraphics[width=10cm]{1.png}
            \caption{Геометрия транзита с отличным от нуля прицельным параметром, b, или что тоже самое,
                отличным от 90\textdegree~наклонением.}
            \label{fig:impact}
        \end{center}
    \end{figure}

    Продолжительность транзита сильно зависит от прицельного параметра.
    Так для круговой орбиты продолжительность транзита будет
    \begin{gather}
        T_{dur}=P\frac{\alpha}{2\pi}, \label{dur}
    \end{gather}
    где $\alpha$~---~это длина дуги орбиты от первого касания до последнего.

    На рис.~\ref{fig:time} показан прямоугольный треугольник, который имеет гипотенузу длины $R_* + R_p$ и
    вертикальную сторону, равную прицельному параметру b, и, следовательно, имеет длину $a \cos i$.
    Третья, горизонтальная, сторона треугольника соединяет положения центра диска планеты в середине транзита
    и четвертого контакта.
    По теореме Пифагора длина этой стороны равна

    \begin{gather}
        l=\sqrt{\left(R_*+R_p \right)^2-a^2\cos^2 i}
    \end{gather}

    Эта величина есть лишь проекция на небо пути планеты, но в нашем случае это допустимое приближение.
    \begin{figure}[h!]
        \begin{center}
            \includegraphics[width=10cm]{4.png}
            \caption{К расчёту продолжительности транзита в случае круговой орбиты.}
            \label{fig:time}
        \end{center}
    \end{figure}

    Следует заметить, что из данного изображения так же можно получить предельные значения прицельного параметра,
    при которых транзит будет происходить

    \begin{gather}
        0\leq a \cos i < R_*+R_p\\
        0\leq \cos i < \frac{R_*+R_p}{a}\approx\frac{R_*}{a}
    \end{gather}
    Но при таком предельном значении диск планеты не полностью будет заходить на диск звезды,
    а значит будет особенным образом изменяться площадь скрываемой части диска звезды.
    Полностью планета будет заходить на диск звезды при

    \begin{gather}
        0\leq \cos i < \frac{R_*-R_p}{a}
    \end{gather}

    Угол $\alpha/2$ в уравнении~\ref{dur} можно приближённо считать как $\arcsin \frac{l}{a}$, тогда
    \begin{gather}
        T_{dur}=\frac{P}{\pi}\arcsin\frac{l}{a}=\frac{P}{\pi}\arcsin \left(\frac{\sqrt{\left(R_*+R_p \right)^2-a^2\cos^2 i}}{a} \right)\\
        T_{dur} \approx \frac{P}{\pi}\arcsin \left(\sqrt{\frac{R_*^2}{a^2}-\cos^2 i} \right) \label{inc}
    \end{gather}

    Последняя выкладка справедлива при $a\gg R_*\gg R_p$.
    Если мы знаем спектральный тип звезды, мы можем вывести приблизительные значения $M_*$ и $R_*$, предположив,
    что они типичны для спектрального типа.
    Тогда продолжительность транзита и орбитальный период достаточны для вывода при помощи
    уравнений~\ref{inc}~и~\ref{kepler} приблизительного наклона орбиты.

    Кроме того, как только наклонение орбиты будет оценено таким образом, мы можем использовать время,
    прошедшее между первым и вторым контактом, чтобы получить вторую оценку радиуса планеты.
    При транзитном входе диск планеты перемещается от первого контакта ко второму контакту.

    \begin{gather}
        T_{1-2}=\frac{P}{\pi}\arcsin \left(\frac{\sqrt{\left(R_*-R_p \right)^2-a^2\cos^2 i}}{a}\right)
    \end{gather}
    Время, необходимое для этого, зависит от наклона орбиты, размеров, продолжительности транзита и периода орбиты,
    дают нам первую и последнюю из этих величин, поэтому ранее известное значение для $R_*$,
    а измерение t1-2 подразумевает значение $R_p$.
    Конечно, глубина транзита уже дала нам оценку соотношения радиуса:


\section{Функция потока}\label{sec:flux}
Поток света от экзопланетной системы сложным зависит от времени.
Во время транзита поток падает, потому что планета блокирует часть звездного света.
Затем поток возрастает, когда в поле зрения появляется дневная сторона планеты.
Поток снова падает, когда планета закрывается звездой.
Поток всей системы отнесённый к потоку звезды можно записать как

\begin{gather}
    f(t)=\frac{F}{F_*}=1+k^2 \frac{I_p(t)}{I_*}-
    \begin{cases}
        k^2 \alpha_1 (t)                    & \text{транзит} \\
        0                                   & \text{вне транзита} \\
        k^2 \frac{I_p(t)}{I_*} \alpha_2 (t) & \text{вторичный транзит}
    \end{cases},
\end{gather}
где F --- поток всей системы; $F_p$ --- поток звезды; $I_p(t)$~---~средняя интенсивность планеты, зависящая от времени;
$I_*$~---~средняя интенсивность звезды, считающаяся постоянной;
k~---~отношение радиуса планеты к радиусу звезды, $k = R_p/R_*$;
$\alpha_1$ и $\alpha_2$~---~безразмерные функции порядка единицы, зависящие от площади перекрытия звездного и
планетарного дисков.
В целом $F_*$ может изменяться во времени из-за вспышек, вращения звездных пятен, вращения приливного горба,
вызванного планетой, или по другим причинам, но для простоты обсуждения мы принимаем её за константу.

Изменения во времени средней интенсивности планеты вызваны изменением освещенной части планетарного диска,
а также любыми изменениями, присущими планетарной атмосфере.
В качестве начального приближения функции $\alpha$ описывают трапеции, а $f(t)$ определяется глубиной транзита $\delta$,
продолжительностью $T$ и длительностью входа или выхода $\tau$.
Для транзитов максимальная потеря света составляет~\cite{winn2014transits}:
\begin{gather}
    \delta_1 \approx k^2 \left( 1-\frac{I_p(t_1)}{I_*} \right),
\end{gather}
но, так как обычно свет с теневой стороны планеты пренебрежимо мал, $\delta_1 \approx k^2$, как отношение площадей дисков.
Для вторичного затмения:
\begin{gather}
    \delta_2 \approx k^2 \frac{I_p(t_2)}{I_*},
\end{gather}

В трапециевидном приближении изменение потока считается линейным во времени при входе и выходе, но это не так.
Во-первых, из-за неравномерного движения звездных и планетных дисков.
Во-вторых, что еще более важно, даже при равномерном движении площадь перекрытия между
дисками не является линейной функцией времени~\cite{Mandel_2002}.
Кроме того, дно кривой блеска не является плоским,
потому что реальные звездные диски не имеют равномерной интенсивности.

\section{Потемнение к краю диска}\label{sec:limb-darkning}
Звезды главной последовательности не имеют твердых поверхностей: их внешние слои состоят из плазмы.
Свет, исходящий от них, исходит из различных глубин звездной атмосферы.
Вероятность выхода фотона, испускаемого в определенном слое атмосферы, зависит от оптической толщины этого слоя.
При заданной частоте, $\nu$, оптическая толщина, $\tau_{\nu}$, является интегралом объёмного показателя ослабления среды,
$k_{\nu}$, умноженным на плотность, $\rho(s)$, вдоль пути, пройденного лучом света
\begin{gather}
    \tau_{\nu} = \int \rho(s) k_{\nu} ds
\end{gather}

Оптическая толщина зависит от частоты излучения, поэтому определенная физическая глубина в атмосфере звезды будет иметь
различную оптическую толщину, в зависимости от рассматриваемой частоты излучения.
Интенсивность света, прошедшего через среду, уменьшается экспоненциально:

\begin{gather}
    \frac{I}{I_0}=e^{-\tau_{\nu}},
\end{gather}
где $I$ --- интенсивность прошедшего света, а $I_0$ --- начальная интенсивность.

Фотон, выходящий из центра звездного диска, движется радиально наружу через звездную атмосферу.
Фотон, появляющийся из любого другого места на звездном диске,
движется под углом $\gamma$ к внешне направленному радиус-вектору.
Для фотонов, испускаемых на глубине h, длина пути через атмосферу звезды равна

\begin{gather}
    s \approx \frac{h}{\cos \gamma} = \frac{h}{\mu},
\end{gather}
где $\mu = \cos \gamma$, мы используем приближение $h \ll R_*$.

Оптическая толщина для данной физической глубины увеличивается по направлению к краю звезды.
Следовательно, звездный диск постепенно тускнеет по направлению к краю.
Более короткая длина пути, пройденного через атмосферу фотонами, движущимися радиально наружу, также означает,
что больше фотонов выходит из более глубоких слоев атмосферы в центре диска.
Как правило, более глубокие слои более горячие, и излучение, испускаемое там приближается к более синему спектру черного тела.
Это означает, что диск постепенно краснеет по направлению к краю, что укореняет нелинейную зависимость потемнения от
длины волны.
Зависимость потемнения звездного края от длины волны сразу же проявляется на многоцветных кривых блеска:
кривая блеска в фильтрах с большой длиной волны, в красной части спектра, имеет более плавно изгибающийся транзитный минимум,
когда же в фильтрах с короткой длиной волны, в синей части спектра, кривая блеска может быть настолько изогнуто,
что нельзя визуально определить точки второго и третьего касания~\cite{Haswell2010}.
Без учёта этого эффекта нельзя получить точные значения параметров системы~\cite{Espinoza}.

\subsection{Однородный диск}\label{subsec:disc}

Существует несколько стандартных моделей потемнения к краю, большая часть которых описываются в работе Мендела и Агола~\cite{Mandel_2002}.
В частности модель однородного диска (Uniform model) описывает транзит как затмение сферической звезды непрозрачной темной сферой,
то есть эта модель не учитывает потемнение к краю и опирается только на геометрические предпосылки задачи.
Для однородного источника отношение затененного к полному потоку равно

\begin{gather}
    F(p,z) = 1 - \lambda (p, z), \\
    \lambda (p, z) =
    \begin{cases}
        0                                                                                                        , & 1+p<z, \\
        \frac{1}{\pi} \left( p^2 k_0 +k_1 - \frac{\sqrt{4 z^{2} - \left( 1+z^{2} - p^{2} \right)^{2}}}{2} \right), & \abs{1-p} < z \leq 1+p, \\
        p^2                                                                                                      , & z \leq 1-p, \\
        1                                                                                                        , & z \leq p-1,
    \end{cases}
\end{gather}
где $d$ --- расстояние между центрами тел, $z=d/R_*$, $p=R_p/R_*$~---~отношение радиусов,
$k_1=\arccos \left[\left( 1-p^2 +z^2 \right) /2z \right]$, $k_0=\arccos \left[ \left( p^2 +z^2 - 1 \right) /2pz \right]$.

\subsection{Quadratic model}\label{subsec:quadratic-model}
Эта модель, так же как и предыдущая, представлена в статье~\cite{Mandel_2002}.
Она рассматривает функцию интенсивности диска звезды для заданной длины волны как квадратичный закон относительно $\mu$:
\begin{gather}
    I(r)=1-\gamma_1 (1-\mu) - \gamma_2 (1-\mu)^2 ,
\end{gather}
где $\gamma_1$ и $\gamma_2$ --- некоторые коэффициенты, характеризующие потемнение к краю, $\gamma_1+\gamma_2<1$.
Впервые такая функция была предложена Копалом~\cite{Kopal} в 1950 году для работы с тесными двойными системами.

Квадратичная модель является продолжением нелинейной модели Кларета~\cite{Claret_2000}.
Мендел и Агол модифицировали её, что позволило перейти от гипергеометрических функций к
более лёгким в вычислении эллиптическим интегралам.

\subsection{QPower2 model}\label{subsec:qpower2-model}
Макстед и Гилл воспользовались~\cite{Maxted_2019} законом с двумя коэффициентами power-2,
предложенным Хестрофером~\cite{Hestroffer}, поскольку тот превосходит прочие законы с двумя коэффициентами,
принятых в литературе в большинстве случаев, в особенности для холодных звёзд~\cite{Morello_2017}.
Функция интенсивности для заданной длины волны выглядит следующим образом
\begin{gather}
    I=1-c (1-\mu ^\alpha),
\end{gather}
где $\mu$ косинус угла между нормалью к поверхности и лучом зрения.

Использование показателя $\mu$, а не коэффициента некоторой степени $\mu$,
позволяет этому закону точно соответствовать форме профиля потемнения к краю звезды,
используя только один дополнительный параметр, в отличие от линейного закона потемнения к краю.

Кривая блеска вычисляется с помощью численного интегрирования.
Время, необходимое для выполнения численного интегрирования отдельного объекта, достаточно мало.
Но, если цель состоит в обнаружении и анализе транзитов в большом количестве высокоточных кривых блеска из таких исследований,
как Kepler, K2 или TESS, время вычислений может быть ограничивающим фактором.
Однако QPower2 позволяет перенести вычисления кривых блеска многих объектов на GPU, что делает данный
алгоритм привлекательным для массового анализа кривых блеска.

%\subsection{Общая модель}\label{subsec:general-model}
%%~\cite{g_2006}.
%
%Представлены простые в использовании аналитические формулы для расчета световых кривых внесолнечных планетарных транзитов.
%Уравнения являются функцией дробных радиусов планеты и родительской звезды,
%наклона орбиты и коэффициентов затемнения конечностей звезды.
%Кривые освещенности могут быть решены для этих параметров в зависимости от точности имеющихся наблюдений.
%Когда кривая радиальной скорости также доступна, как это обычно бывает для обеспечения природы системы,
%можно определить массы, радиусы и среднюю плотность как звезды, так и планеты.
%Уравнения справедливы для любой степени затемнения конечностей, а также для любого типа транзита.
%Случаи эксцентрических орбит, третьего света или
%ненулевой относительной светимости планеты могут быть легко приняты во внимание.
%Основное предположение состоит в том, что проекции как звезды,
%так и планеты на плоскость неба хорошо представлены круглыми дисками.
%Также обсуждаются последствия в случае, если это предположение неверно.
%Показаны практические применения, начиная с кривой освещенности фотометрически обнаруженной планеты OGLE-TR-113,
%полученной с помощью наземного телескопа.
%В качестве второго примера приведены результаты исследования кривой освещенности,
%полученной для прохождения гигантской планеты в HD 209458 с помощью космического телескопа Хаббла.
%Кратко обсуждаются процедуры для получения наилучших параметров соответствия.